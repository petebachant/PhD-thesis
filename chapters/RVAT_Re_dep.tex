\chapter{Reynolds number effects on the performance and near-wake of a high
solidity cross-flow turbine} \label{chap:Re-dep}

The use of scaled physical models is common practice in engineering and science
to both assess designs against their requirements and evaluate (validate) the
methods---often computational---use to predict their behavior. Scaling models
helps reduce cost, but relevant physical parameters are then often mismatched.


\section{Reynolds number effects on foil performance}

The performance of a foil is governed by its boundary layer dynamics, namely the
processes of transition from a laminar to turbulent boundary layer, and flow
separation, where the suction side boundary layer encounters a reversed flow due
to an adverse pressure gradient.


\section{Reynolds number effects on wakes}

To obtain a first order prediction of how the wake of the turbine might change
with $Re$, we will simplify the Navier--Stokes equations and look for scaling
behavior in the wake of a generic momentum sink.