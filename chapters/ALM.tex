\chapter{Developing an actuator line model for cross-flow turbines}

After getting a sense for how two different turbines perform, and how they
affect the flow in which they are placed, the goal is to now develop a way to
predict these effects without resorting to blade-resolved Navier--Stokes CFD.

The ultimate goal of the actuator line model is to drive down the computational
costs of simulating turbines by negating the need for complicated meshing, and
the subsequent boundary layer resolution.

The ALM is based on the classical blade element theory, developed by
\todo[inline]{Find citation for creator of blade element theory}
and subsequently first implemented by Sorensen and Shen \cite{Sorensen2002}.

\section{Model inputs}

% Not multiple Re coefficient tables!

\section{Blade element discretization}

In the ALM, a turbine is a collection of actuator lines, which themselves are
collections of actuator line elements (ALEs). The position of each ALE is a
point in space indicating the quarter-chord location of the element
cross-section. The element is further defined by its chord direction vector,
chord length, span direction vector, span length, and velocity vector.

\section{Force projection}

After the force on the ALE from the flow is calculated, it is then projected
back onto the flow field as a source term in the momentum equation. To avoid
instability due to sharp gradients, the source term is tapered from its maximum
value away from the element location by means of a spherical Gaussian function.
The width of this function is
\todo[inline]{Add final width of Gaussian projection}

\section{Unsteady aerodynamics}

In the context of a turbine---especially a cross-flow turbine---the actuator
lines will encounter unsteady conditions, both in their angle of attack and
relative velocity. These conditions necessitate the use of unsteady aerodynamic
models to augment the static foil characteristics. Furthermore, the angles of
attack encountered by a CFT blade will often be high enough to encounter dynamic
stall. It is therefore necessary to model both unsteady attached and detached
flow to obtain accurate loading predictions.

Leishman and Beddoes developed a semi-empirical model for unsteady aerodynamics
and dynamic stall, which is derived from the phenomenology of the physics
instead rather than pure empiricism \cite{Leishman1989}.

\section{Flow curvature corrections}

The rotating blades of a cross-flow turbine will have varying angle of attack
along their chords for any given azimuthal location due to the circular path.
This makes it difficult to define a single angle of attack for use in the static
coefficient lookup tables. Furthermore, this effect is more pronounced for high
solidity ($c/R$) turbines.


\section{Reynolds number corrections}
As seen in Chapter~\ref{chap:Re-dep}

\section{Effects on turbulence modeling}

Conventional blade element simulations use either momentum or vortex methods to
solve for the incident flow field, and these methods do not model the effects of
turbulence. With the actuator line model, there is the opportunity to improve
the physical realism by not only adding a source to the momentum equations, but
also to the turbulence model equations.

James et al. implemented an actuator disk model in a RANS model with a
$k$--$\epsilon$ closure, which ``injected'' $k$ and $\epsilon$ from the actuator
disks to more realistically simulate the turbine's turbulent wake and enhance
momentum transport \cite{James2010}. However, to the author's knowledge
injecting turbulence quantities has never been done in an actuator line model.

In this case, we seek to inject turbulence dependent on blade loading, both
for the $k$--$\epsilon$ RANS model and the
\todo[inline]{Pick LES model to work with.}

\section{Software implementation}

\section{Results}

\subsection{RANS}

\subsection{LES}

\section{Computational cost}
% Compare with the blade-resolved RANS