\chapter{Development and evaluation of an actuator line model for cross-flow
    turbines}\label{chap:ALM}

Despite its apparent---but not completely certain---effectiveness for predicting
the performance and wake of a single turbine, blade-resolved CFD presents a huge
computational expense since it must resolve fine details of the blade boundary
layers, and must be performed in three dimensions, which will preclude its use
for array analysis until the availability of computing power increases
sufficiently. It is therefore necessary to explore simpler models that can
predict the turbine loading and flow field with acceptable fidelity, but that
are economical enough to not require high performance computing, at least for
individual devices.

Very fast solution times can be obtained with blade element momentum (BEM)
models, such as the double multiple streamtube (DMST) approached developed by
Paraschivoiu~\cite{Para1988}. However, their poor performance for high solidity
turbines \cite{Joo2015}, makes them less attractive, particular for marine
hydrokinetic applications. Their description of the flow field is very crude as
well, simply computing the momemtum contained within discretized streamtubes,
which cannot resolve any of the complicated flow structures shed by the rotor.

Blade element based vortex line methods can be solved with reasonable
computational effort---ranging from an expense close to BEM up to that of 2-D
blade-resolved CFD. The cost is a function of the number of vortex elements,
which for free vortex methods increases at each time step, slowing down the
calculation as it marches forward in time. Vortex methods can resolve more flow
details than BEM. However, since the flow is governed by the Laplace equation,
nonlinear effects such as the turbulent transport will not be included, which we
have determined in experiments to be the same order of magnitude as the mean
vertical advection.

It is therefore desirable to retain a Navier--Stokes description of the flow
field and use an actuator-type model for parameterizing the turbine loading,
which dramatically drives down computational expense by removing the need for
complicated meshes with many fine cells to resolve the boundary layers near the
blade surfaces. As shown in Chapter~\ref{chap:RVAT-baseline}, the conventional
uniform actuator disk is not a good candidate for a cross-flow turbine wake
generator, never mind the fact that it does not typically compute performance
predictions. There are then two actuator methods that use blade element theory
to compute blade loading and therefore generate a wake that more closely
resembles that of an actual turbine. These are the actuator cylinder or
swept-surface model (ASSM) and the actuator line model (ALM). The ASSM solves
for the average blade loading along its path and applies this as a constant body
force term in the momentum equation. The ALM takes a similar approach but is an
unsteady method, resolving the blade element locations in time.

The ALM, originally developed by Sorensen and Shen \cite{Sorensen2002}, has
become popular for modeling axial-flow or horizontal-axis turbines, and has been
shown in blind tests to be competitive with blade-resolved
CFD~\cite{Krogstad2013, Pierella2014}. The ALM combined with large eddy
simulation has become the state-of-the-art for modeling entire wind
farms~\cite{Archer2013, Churchfield2012, Sorensen2015, Fleming2013,
    Fleming2014}. Like other blade element techniques, the effectiveness of the ALM
for AFTs is in part due to the quasi-steady nature of the flow in the blade
reference frame, and the relatively rare occurrence of stall.

The ASSM and ALM were implemented to model a very low Reynolds number 2-D
cross-flow turbine experiment in a flume using large eddy simulation
(LES)~\cite{Shamsoddin2014}. Performance predictions for this case were not
reported, but the ALM was shown to be more effective at postdicting the wake
characteristics measured in the experiments by Brochier et
al.~\cite{Brochier1986}.

It is therefore proposed that an actuator line model may be the optimal
combination of high-fidelity flow modeling that includes performance
predictions, but with reduced computational expense. Here we will develop and
evaluate the effectiveness of an ALM for cross-flow turbines, implemented in
both RANS and LES models.


\section{Theory}

The actuator line model is based on the classical blade element theory combined
with a Navier--Stokes description of the flow field. The ALM treats turbine
blades as lines of blade elements, for which 2-D profile lift and drag
coefficients are given. For each blade element, relative flow velocity and angle
of attack are computed by adding the vectors of relative blade motion and the
local fluid velocity. The blade lift and drag forces are calculated as
\begin{equation}
    F_l = \frac{1}{2} \rho A_\mathrm{elem} C_l |\vec{U}_\mathrm{rel}|^2,
\end{equation}
and
\begin{equation}
    F_d = \frac{1}{2} \rho A_\mathrm{elem} C_d |\vec{U}_\mathrm{rel}|^2,
\end{equation}
where $\rho$ is the fluid density, $A_\mathrm{elem}$ is the blade element
planform area (span $\times$ chord), $\vec{U}_\mathrm{rel}$ is the local
relative velocity, and $C_l$ and $C_d$ are the sectional lift and drag
coefficients, linearly interpolated from a table per the local angle of attack.
The forces are then projected onto the rotor coordinate system to calculate
torque, overall drag, etc. Forces from the turbine shaft and blade support
struts are computed in a similar way. After the force on the actuator lines from
the flow is computed, it is then added to the Navier--Stokes equations as a body
force or momentum source (per unit density, assuming incompressible flow):
\begin{equation}
    \frac{\mathrm{D} \vec{u}}{\mathrm{D} t} = - \frac{1}{\rho} \nabla p + \nu
    \nabla^2 \vec{u} + F_\mathrm{turbine}.
\end{equation}


\section{Blade element discretization}

In the ALM, a turbine is a collection of actuator lines, which themselves are
collections of actuator line elements (ALEs). The position of each ALE is a
point in space indicating its quarter-chord location. The element is further
defined by its chord direction vector, chord length, span direction vector, span
length, and velocity vector.

An actuator line is created from defined geometry points, between which ALE
parameters are interpolated linearly. This way, an actuator line can be defined
by fewer geometry points than element locations. For example, an AL with
straight planform boundaries---e.g. a straight or tapered wing---only needs two
geometry points to be fully defined. Figure~\ref{fig:AL-geom} shows and example
schematic of a tapered actuator line with three geometry points at the
half-chord locations and six total elements. Note that the middle geometry point
is technically redundant, but is shown for illustrative purposes. For actuator
lines that represent bluff bodies, e.g., shafts, the chord mount offset is set
to 1/4, such that the element location is centered along the line of geometry
points.

\begin{figure}
    \centering

    \includegraphics[width=0.9\textwidth]{alm-geometry}

    \caption{Actuator line geometry. Filled circles indicate geometry points
        whereas squares indicate actuator line element locations.}

    \label{fig:AL-geom}
\end{figure}


\section{Determining inflow velocity}

In momentum methods, the inflow velocity is determined by solving for the axial
and angular induction factors \cite{Manwell2002}. However, using Navier--Stokes
methods, it is somewhat unclear how to calculate the velocity vector used to
compute the angle of attack and relative velocity, though we have access to much
more information about the flow. Sorensen and Shen used an actuator line
element's position to determine the inflow velocity for an axial-flow turbine
\cite{Sorensen2002}. Similarly, Shamsoddin and Porte-Agel use the velocity at a
blade element's location in their actuator line simulation of a vertical-axis
turbine using LES \cite{Shamsoddin2014}. NREL's SOWFA ALM for OpenFOAM also uses
the velocity at the actuator line element location for computing inflow velocity
and local angle of attack with no corrections \cite{Churchfield2013}.

Schito and Zasso developed an effective velocity model (EVM) for computing
actuator forces in Navier--Stokes simulations \cite{Schito2014}. Their EVM
proposes that inflow velocity should be sampled along a line perpendicular to
the mean relative flow direction. They ultimately chose a the line to be 1.5
chord lengths upstream of the actuator point (i.e. quarter-chord location). A
sampling line was chosen to be 5 times the local mesh cell length. Finally, an
angle of attack correction is proposed
\begin{equation}
    \Delta \alpha = \frac{c}{M} (1.2553 - 0.0552 C_d) C_l,
    \label{eq:EVM-dalpha}
\end{equation}
where $c$ is the chord length, $M$ is the mesh size, and $C_l$ and $C_d$ are the
lift and drag coefficients, respectively. Note that the constants in
Equation~\ref{eq:EVM-dalpha} were determined from a calibration with 2-D
blade-resolved CFD for a NACA 0012 foil and are not assumed to be universal.

The EVM, despite showing robustness for its chosen validation case,
unfortunately involves determination of two unknown tuning parameters. To avoid
the additional effort and uncertainty in determining these, the inflow velocity
was sampled at the element quarter-chord location using OpenFOAM's
\texttt{interpolationCellPoint} class, which provides a linear weighted
interpolation using cell values. This algorithm helps keep the sampled velocity
``smooth'' compared with using the cell values themselves, especially when
elements are moving in space as they are in a turbine, since meshes will likely
have a cell size on the same order as the chord length, and will move on the
order of one cell length per time step.


\section{Static foil coefficient data}

Static input foil coefficient data were taken from Sheldahl and
Klimas~\cite{Sheldahl1981}---a popular database developed for CFTs, which
contains values over a wide range of Reynolds numbers. NACA 0021 coefficients
were used for both turbines, despite the fact that the UNH-RVAT is constructed
from NACA 0020 foils, as a NACA 0020 dataset is not available---it is assumed
the small difference in foil thickness is negligible. Since pitching moment data
were only available at limited Reynolds numbers, two datasets were used: The
lowest for $Re_c \leq 3.6 \times 10^5$ and highest $Re_c \geq 6.8 \times 10^5$.
For each actuator line element, blade chord Reynolds number is computed based on
the sampled inflow velocity, and the static coefficients are then interpolated
linearly within the database.


\section{Force projection}

After the force on the ALE from the flow is calculated, it is then projected
back onto the flow field as a source term in the momentum equation. To avoid
instability due to sharp gradients, the source term is tapered from its maximum
value away from the element location by means of a spherical Gaussian function.
The width of this function $\eta$ is controlled by a single parameter
$\epsilon$, which is then multiplied by the actuator line element force and
imparted on a cell with distance $| \vec{r} |$ from the actuator line element
quarter chord location:
\begin{equation}
    \eta = \frac{1}{\epsilon^3 \pi^{3/2}} \exp
    \left[ - \left( \frac{| \vec{r} |}{\epsilon} \right)^2 \right].
    \label{eq:projection}
\end{equation}

Troldborg~\cite{Troldborg2008} proposed that the Gaussian width should be set to
twice the local cell length $\Delta x$ in order to maintain numerical stability.

Jha et al.~\cite{Jha2014} provides guidelines for choosing a projection
width.

Schito and Zasso \cite{Schito2014} found that a projection $\epsilon$ equal to
the local mesh length was optimal.

Martinez-Tossas and Meneveau \cite{Martinez-Tossas2015b} used a 2-D potential
flow analysis to determine that the optimal projection width for a lifting
surface is 14--25\% of the chord length. The width due to the wake caused by the
foil drag force was recommended to be on the order of the momentum thickness
$\theta$, which for a bluff body or foil at large angle of attack is related to
the drag coefficient ($O(1)$) by \cite{TennekesAndLumley}
\begin{equation}
    C_d = 2 \theta / l,
    \label{eq:mom-thickness}
\end{equation}
where $l$ is a reference length, e.g., diameter for a cylinder or chord length
for a foil.

Using these guidelines, three Gaussian width values were determined: one
relative to the chord length, one to the mesh size, and one to the momentum
thickness due to drag force. Each three were computed for all elements at each
time step, and the largest was chosen for the force projection algorithm. Using
this adaptive strategy, fine meshes could benefit from the increased accuracy of
more concentrated momentum sources, and coarse meshes would be protected from
numerical instability.

The Gaussian width due to mesh size $\epsilon_{\mathrm{mesh}}$ was determined
locally on an element-wise basis by estimating the size of the cell containing
the element as
\begin{equation}
    \Delta x \approx \sqrt[3]{V_\mathrm{cell}},
\end{equation}
where $V_\mathrm{cell}$ is the cell volume. To account for the possibility of non-unity aspect ratio cells, an additional factor $C_\mathrm{mesh}$, 2.0 by default, was introduced, giving
\begin{equation}
    \epsilon_{\mathrm{mesh}} = 2C_\mathrm{mesh} \Delta x.
\end{equation}


\section{Unsteady effects}

In the context of a turbine---especially a cross-flow turbine---the actuator
lines will encounter unsteady conditions, both in their angle of attack and
relative velocity. These conditions necessitate the use of unsteady aerodynamic
models to augment the static foil characteristics, both to capture the time
resolved response of the attached flow loading and effects of flow acceleration,
also know as added mass. Furthermore, the angles of attack encountered by a CFT
blade will often be high enough to encounter dynamic stall (DS). It is therefore
necessary to model both unsteady attached and detached flow to obtain accurate
loading predictions.


\subsection{Dynamic stall}

Dynamic stall is encountered when the blade angle of attack changes rapidly in
time and exceeds a certain threshold, often near the static stall angle
\cite{McCroskey1981}. The stall is characterized by an initial increase in lift
beyond static values as a vortex is shed from the foil's leading edge, after
which a drop in lift and large nose-down pitching moment occurs as the vortex is
advected downstream. As the angle of attack drops below the critical value, flow
reattaches, closing the so-called hysteresis loop. Dynamic stall has been shown
to be a significant positive contributor to performance in CFTs \cite{Para2002,
    Urbina2013}, therefore an accurate model is key.

DS models were first developed to improve predictive capability for helicopter
rotors, on which DS has significant effects on maneuverability and operational
envelope \cite{Bousman2000}. A summary of dynamic stall models developed for
helicopter rotors is presented in \cite{Leishman2006}. The simplest dynamic
stall models rely on semi-empirical correlations, e.g., the Gormont model
\cite{Gormont1973}, developed at the Boeing--Vertol Company. Several variants of
the Gormont model were developed for vertical-axis wind turbines, with varying
degrees of success; a summary is presented in \cite{Para2002}.

Leishman and Beddoes (LB) \cite{Leishman1989} developed a semi-empirical model
for unsteady aerodynamics and dynamic stall, which is derived from the
phenomenology of the physics instead rather than pure empiricism. Beddoes then
updated the model to the so-called third generation or ``3G'' version
\cite{Beddoes1993}. The LB DS models can be summarized conceptually based on the
following principles:
\begin{itemize}
    \item Dynamic conditions cause a time lag in effective angle of attack and
    lift force.

    \item Separation is determined by the Kirchoff flow approximation, which is
    also used to parameterize the normal force coefficient table based on the
    trailing edge separation point. This separation point also encounters a time
    lag.

    \item The separation initiates a vortex shedding cycle that causes an
    overshoot and subsequent undershoot in lift before returning to an attached
    flow condition.
\end{itemize}

Sheng et al. \cite{Sheng2008} developed an LB DS model variant targeted at low
Mach numbers. This model, along with the original and 3G LB DS model variants,
was tested for its effectiveness in cross-flow turbine conditions by Dyachuk et
al.~\cite{Dyachuk2014}, who concluded that the Sheng et al. variant results
matched most closely with experiments. In a similar study, the Sheng et al.
model also faired better than the Gormont model \cite{Dyachuk2015}, which
inspired its adoption here for the ALM.

Before the dynamic stall subroutine is executed, the static profile data for
each element is interpolated linearly based on local chord Reynolds number. The
profile data characteristics---static stall angle, zero-lift drag coefficient,
and separation point curve fit parameters---are then recomputed each time step
such that the effects of Reynolds number on the static data are included.

Inside the ALM, angle of attack is sampled from the flow field rather than
calculated based on the geometric angle of attack. Therefore, the implementation
of the LB DS model was such that the equivalent angle of attack
$\alpha_\mathrm{equiv}$ was taken as the sampled rather than the lagged
geometric value. A similar implementation was used by Dyachuk et al.
\cite{Dyachuk2015a} inside a vortex model.


\subsection{Added mass}

A correction for added mass effects, or the effects due to accelerating the
fluid, was taken from Strickland et al.~\cite{Strickland1981}, which was
derived by considering a pitching flat plate in potential flow. In the blade
element coordinate system, the normal and chordwise (pointing from trailing to
leading edge, which is opposite the $x$-direction used by Strickland \emph{et
    al.}) coefficients due to added mass are
\begin{equation}
    C_{n_\mathrm{AM}} = -\frac{\pi c \dot{U_n}}{8 | U_\mathrm{rel} |^2},
\end{equation}
and
\begin{equation}
    C_{c_\mathrm{AM}} = \frac{\pi c \dot{\alpha} U_n }{8 | U_\mathrm{rel} |^2},
\end{equation}
respectively, where $U_n$ is the normal component of the relative velocity, and
dotted variables indicate time derivatives, which were calculated using a simple
first order backward finite difference. Similarly, the quarter-chord moment
coefficient due to added mass was calculated as
\begin{equation}
    C_{m_\mathrm{AM}} = -\frac{C_{n_\mathrm{AM}}}{4}
        - \frac{U_n U_c}{8 | U_\mathrm{rel} |^2},
\end{equation}
where $U_c$ is the chordwise component of relative velocity. Note that the
direction of positive moment is ``nose-up,'' which is opposite that used by
Strickland et al..

The normal and chordwise added mass coefficients translate to lift and drag
coefficients by
\begin{equation}
    C_{l_\mathrm{AM}} = C_{n_\mathrm{AM}} \cos \alpha + C_{c_\mathrm{AM}} \sin
    \alpha,
\end{equation}
and
\begin{equation}
    C_{d_\mathrm{AM}} = C_{n_\mathrm{AM}} \sin \alpha - C_{c_\mathrm{AM}} \cos
    \alpha,
\end{equation}
respectively. The added mass coefficients were then added to those calculated by
the dynamic stall model.


\section{Flow curvature corrections}

The rotating blades of a cross-flow turbine will have non-constant chordwise
angle of attack distributions due to their circular paths---producing so-called
flow curvature effects~\cite{Migliore1980}. This makes it difficult to define a
single angle of attack for use in the static coefficient lookup tables.
Furthermore, this effect is more pronounced for high solidity ($c/R$) turbines.
Two different flow curvature corrections were considered: one by Goude
\cite{Goude2012} and one by Mandal and Burton \cite{Mandal1994}.

The Goude correction is derived by considering a flat plat moving along a
circular path in potential flow, for which the effective angle of attack
including flow curvature effects is given by
\begin{equation}
    \alpha = \delta + \arctan \frac{V_\mathrm{abs} \cos(\theta_b -
        \beta)}{V_\mathrm{abs} \sin(\theta_b - \beta) + \Omega R} - \frac{\Omega
        x_{0r}c}{V_\mathrm{ref}} - \frac{\Omega c}{4 V_\mathrm{ref}},
    \label{eq:Goude-curvature}
\end{equation}
where $\delta$ is the blade pitch angle, $V_\mathrm{abs}$ is the magnitude of
the local inflow velocity at the blade, $\theta_b$ is the blade azimuthal
position, $\beta$ is the direction of the inflow velocity, $\Omega$ is the
turbine's angular velocity, $R$ is the blade element radius, $x_{0r}$ is a
normalized blade attachment point along the chord (or fractional chord distance
of the mounting point from the quarter-chord), $c$ is the blade chord length,
and $V_\mathrm{ref}$ is the reference flow velocity for calculating angle of
attack.

In the actuator line model, each element's angle of attack is calculated using
vector operations, which means the first two terms in
Equation~\ref{eq:Goude-curvature} are taken care of automatically since each
element's inflow velocity, chord direction, and element velocity vectors are
tracked. Therefore, the last two terms in Equation~\ref{eq:Goude-curvature} were
simply added to the scalar angle of attack value. Note that for a cross-flow
turbine, this correction effectively offsets the angle of attack, which
therefore increases its magnitude on the upstream half of the blade path, and
decreases its magnitude on the downstream half, where the angle of attack is
negative.

The Mandal--Burton flow curvature correction assumes that since the blade is
encountering a curvilinear flow, it can be treated as having virtual camber.
They introduce a factor to describe the variation of angle of attack from the
leading to trailing edge
\begin{equation}
    \Delta \alpha = \alpha_\mathrm{TE} - \alpha_\mathrm{LE},
    \label{eq:Mandal-Burton-alpha-diff}
\end{equation}
where TE and LE subscripts denote the values of angle of attack at the trailing
and leading edge, respectively. Calculating these values for an actuator line
element can be done by tracking the leading and trailing edge locations and
velocities, then performing the same vector arithmetic used to calculate the
quarter-chord angle of attack.

An incidence correction factor
\begin{equation}
    \alpha_c = \arctan \left( \frac{1 - \cos (\Delta \alpha / 2)}{\sin (\Delta
        \alpha / 2)} \right)
    \label{eq:Mandal-Burton-alpha-corr}
\end{equation}
is introduced and added to the uncorrected angle of attack. Like the Goude
model, $\alpha_c$ is positive on the upstream half of the turbine rotation and
negative on the downstream half.

It was determined that the Goude model performed best with respect to matching
the near-wake characteristics measured in the experiments, while the performance
predictions from both flow curvature models were relatively close to each other.
Thus, the Goude model was used for all simulations reported here.


\section{End effects}

Helmholtz's second vortex theorem states that vortex lines may not end in a
fluid, but must either form closed loops or extend to boundaries. Consequently
the lift distribution due to the bound vortex from foils of finite span must
drop to zero at the tips.

Glauert\cite{Glauert1935} used Prandtl's lifting line theory \cite{Prandtl1927}
to develop a tip loss correction factor $F$ to be applied to the local lift of
an axial-flow rotor blade element:
\begin{equation}
    F = \frac{2}{\pi} \cos^{-1} \left[ \exp \left( - \frac{N (R-r)}{2R \sin
        \phi_R} \right) \right],
\end{equation}
where $F$ is the correction factor applied to the local lift, $N$ is number of
blades, $R$ is the rotor radius, $r$ is the local blade element radius, and
$\phi_R$ is the flow angle at the tip.

The Glauert correction was further refined for horizontal-axis wind turbines by
Shen et al. \cite{Shen2005a} as
\begin{equation}
    F_1 = \frac{2}{\pi} \cos^{-1} \left[ \exp \left( -g \frac{N (R-r)}{2R \sin
    \phi_R} \right) \right],
\end{equation}
where $g$ is an additional function depending on the tip speed ratio and two
tuning constants:
\begin{equation}
    g = \exp [ -c_1 (B \lambda - c_2) ].
\end{equation}

Despite their success for the blade element analysis of axial-flow rotors, these
corrections both depend on rotor parameters---tip speed ratio, number of blades,
element radius, tip flow angle---that do no necessarily translate directly to
the geometry and flow environment of a cross-flow rotor. Therefore, a more
general end effects model was sought.

From Prandtl's lifting line theory, the geometric angle of attack $\alpha$ of a
foil with an arbitrary circulation distribution can be expressed as a function
of nondimensional span $\theta$ as \cite{Anderson2001}
\begin{equation}
    \alpha (\theta) = \frac{2S}{\pi c (\theta)}
    \sum_1^N A_n \sin \theta
    + \sum_1^N n A_n \frac{\sin n \theta}{\sin \theta}
    + \alpha_{L = 0}(\theta),
    \label{eq:lifting-line}
\end{equation}
where $S$ is the span length, $c(\theta)$ is the chord length, and $N$ is the
number of locations or elements sampled along the foil. This relationship can be
rearranged into a matrix equation to solve for the unknown Fourier coefficients
$A_n$,
\begin{equation}
    [\alpha_m ] - \alpha_{L=0} = [D_{mn}][A_n],
\end{equation}
where
\begin{equation}
    D_{mn} = \sum_1^N \left[ \frac{2b}{\pi c_m} \sin n \theta_m + n \frac{\sin n
        \theta_m}{\sin \theta_m} \right].
\end{equation}

With the Fourier coefficients, the circulation distribution can be calculated as
\begin{equation}
    \Gamma (\theta) = 2SU_\infty \sum_1^N A_n \sin n \theta,
\end{equation}
which, via the Kutta--Joukowski theorem, provides the lift coefficient
distribution
\begin{equation}
    C_l(\theta) = \frac{-\Gamma (\theta)}{\frac{1}{2} c U_\infty}.
\end{equation}

We can therefore compute a correction function based on the normalized spanwise
lift coefficient distribution
\begin{equation}
    F = C_l(\theta)/C_l(\theta)_{\max},
\end{equation}
which will be in the range $[0, 1]$, similar to the Glauert corrections, but
does not contain rotor parameters.

The correction functions for the standard Glauert and lifting line methods are
plotted for comparison in Figure~\ref{fig:end-effects}. Function values were
computed using the UNH-RVAT geometry, $\lambda = \lambda_0 = 1.9$, and a 20
degree tip angle of attack. Both end effects corrections produce similar results
for this specific case, which gives confidence that the lifting line method will
produce reasonable results without the uncertainty of translating axial-flow to
cross-flow rotor parameters.

\begin{figure}
    \centering

    \includegraphics[width=0.75\textwidth]{end-effects_rvat-20-deg}

    \caption{End effect correction function values for the Glauert and lifting
        line methods using UNH-RVAT parameters at 20 degrees tip angle of attack and
        $\lambda=1.9$.}

    \label{fig:end-effects}
\end{figure}


\section{Software implementation}

The USA National Renewable Energy Laboratory (NREL) has developed and released
an actuator line modeling library, SOFWA~\cite{Churchfield2014b}, for simulating
horizontal-axis wind turbine arrays using the OpenFOAM finite volume CFD
library. OpenFOAM is free, open-source, and widely used throughout industry and
academia. Though SOWFA is also open-source, its procedural style would have
required significant effort and duplicate code to adapt for cross-flow turbines.
Thus, a new and more general ALM library was developed from the ground up that
could model both cross- and axial-flow turbines, as well as standalone actuator
lines. The actuator line model developed here, dubbed \textit{turbinesFoam}
\cite{Bachant2016-turbinesFoam}, was also written as an extension library for
OpenFOAM, and was developed freely and openly from its inception in order to
increase community engagement and research efficiency.

\textit{turbinesFoam} was written in OpenFOAM's style, using OpenFOAM's
\texttt{fvOptions} framework for adding source terms to equations at
runtime---see Listing~\ref{lst:fvOptions} for an example implementation within
the Navier--Stokes' momentum equation. Using the \texttt{fvOptions} framework
allows the CFT-ALM to be added to many of the solvers included in OpenFOAM,
meaning it can be readily used with RANS or LES, multiphase models (e.g., for
simulating the free surface in MHK installations), and even with heat transfer.
This is in contrast to SOWFA's implementation, which requires custom flow
solvers to be developed and maintained.

\begin{lstlisting}[float,language=C++,caption=Adding source terms to the momentum equation in OpenFOAM.,label=lst:fvOptions]
    tmp<fvVectorMatrix> UEqn
    (
        fvm::ddt(U)
        + fvm::div(phi, U)
        + turbulence->divDevReff(U)
        ==
        fvOptions(U)
    );
\end{lstlisting}

OpenFOAM and \textit{turbinesFoam} are written in the C++ programming language,
which follows the object oriented programming paradigm. This characteristic
helped modularize the ALM code for increased readability and reuse. In
\textit{turbinesFoam}, a turbine is a software object that is composed of
actuator line objects, which themselves are composed of actuator line element
objects. Structuring the code this way allows isolation and reuse of the
functionality of individual components. For example, the actuator line object
was written such that is could be used outside the turbine context to ensure is
produces the correct forcing, without adding the complexity of rotation, other
actuator lines, etc. that would be present in a turbine rotor. The very same
actuator line objects can be used in both axial-flow and cross-flow rotors,
without having to copy code from one to the other. In contrast, the actuator
line model in \textit{SOWFA} uses a single software object to represent an
entire array of turbines, which necessitates iterating through many nested lists
down to the element level, which can be confusing to read.

OpenFOAM's data structures are designed to be inherently parallel via message
passing interface (MPI). By working within the library infrastructure the ALM
code was easily parallelized, which will facilitate its deployment on high
performance computing clusters for large flow simulations.

Since all applications are run from a command line and all input data is text
based, automation and integration with other tools is relatively
straightforward. Future enhancements could include coupling with software for
generating static foil data, e.g., XFOIL or other OpenFOAM solvers, turbine
controller models, structural analysis codes, and optimization tools, e.g.,
SNL's DAKOTA \cite{AdamsBaumanBohnhoffEtAl2009}, for both individual turbines
and array layouts.


\section{Results}

Both the high solidity UNH-RVAT and medium-low solidity RM2 turbines were
simulated using a $k$--$\epsilon$ Reynolds-averaged Navier--Stokes (RANS)
turbulence model. These rotors provide diverse parameters, which helped evaluate
the robustness of the ALM. The simulations were performed inside a domain
similar in size to that used in Chapter~\ref{chap:CFD} to mimic the tow tank
(3.66 m wide, 2.44 m tall, 1.52 m upstream and 2.16 m downstream), with similar
boundary conditions (1 m/s inflow, no-slip walls and bottom, and a rigid slip
for the top). Simulations were run for a total of 6 seconds, with the latter
half used to calculate performance and wake statistics. Pressure--velocity
coupling for the momentum equation was achieved using the PISO (pressure
implicit splitting of operators) method. A slice of the mesh in the $x$--$y$
plane is shown in Figure~\ref{fig:ALM-mesh}.

\begin{figure}
    \centering

    \includegraphics[width=0.8\textwidth]{alm-mesh}

    \caption{$x$--$y$ planar slice of the mesh used for the ALM RANS
        simulations.}

    \label{fig:ALM-mesh}
\end{figure}

Similar numerical settings were used for each turbine as well. The Sheng et al.
DS model was used with the default coefficients given in \cite{Sheng2008}, and
the Goude flow curvature correction was employed. A second order backward
difference was used for advancing the simulation in time, and second order
linear schemes were used for the majority of the terms' spatial discretizations.
The only major difference between the two simulation configurations was that the
end effects model was deactivated for the RM2, since it reduced $C_P$ far below
the experimental measurements. This modification is at least consistent with the
RM2 blades' higher aspect ratio and tapered planform. The number of elements per
actuator line was set to be approximately equal to the total span divided by the
Gaussian force projection width $\epsilon$. Case files for the UNH-RVAT and RM2
ALM simulations are available from XXX and XXX, respectively. \todo[inline]{Cite
    ALM case files.}

The same foil coefficient data were used for all simulations---those for a NACA
0021 as reported by Sheldahl and Klimas \cite{Sheldahl1981}. Each rotor's shaft
was assumed to have a drag coefficient $C_d = 1.1$, and the blade support strut
end element drag coefficients were set to 0.05, to approximate the effects of
separation in the corners of the blade--strut connections.

Since the ALM is intended to be an engineering tool when coupled with RANS, it
was assumed that information about tip speed ratio due to control details would
not be known a priori, and was excluded, unlike the 3-D blade-resolved cases in
Chapter~\ref{chap:CFD}. Note that a systematic investigation of the effects of
sinusoidal $\lambda$ was not undertaken, but for the UNH-RVAT a one to two
percentage point increase in $C_P$ was observed when running with similar
parameters as the blade-resolved simulation.

Compared to 3-D blade-resolved RANS, the ALM can solve a standalone turbine case
on the order of CPU minutes per second of simulated time versus 1,000 CPU hours
per second---a savings of 4 orders of magnitude.


\subsection{Verification}

Verification for sensitivity to spatial and temporal grid resolution was
performed for both the UNH-RVAT and RM2 RANS cases at their optimal tip speed
rations, the results from which are plotted in
Figure~\ref{fig:RVAT-ALM-verification} and
Figure~\ref{fig:RM2-ALM-verification}, respectively. Similar to the verification
strategy employed in Chapter~\ref{chap:CFD}, the mesh topology was kept
constant, and the resolution was scaled proportional to the number of cells in
the $x$-direction $N_x$ for the base hexahedral mesh. Both models displayed low
sensitivity to the number of time steps per revolution. Spatial grid dependence,
however, was more important.

Final spatial grid resolutions were chosen as $N_x=48$ for both the UNH-RVAT and
RM2 cases. Time steps were chosen as $\Delta t = 0.01$ and $\Delta t = 0.005$
seconds for the UNH-RVAT and RM2 respectively, which correspond to approximately
200 steps per revolution. The chosen values should provide $C_P$ predictions
within one percentage point of the true solution, which is on the order of the
expanded uncertainty of the experimental measurements. Note that for computing
performance curves, the number of steps per revolution was kept constant, i.e.,
the time step was adjusted to $\Delta t = \Delta t_0 \lambda_0 / \lambda$.

\begin{figure}
    \centering

    \includegraphics[width=0.85\textwidth]{RVAT-ALM_verification}

    \caption{Temporal (left) and spatial (right) grid resolution sensitivity
        results for the UNH-RVAT ALM RANS model.}

    \label{fig:RVAT-ALM-verification}
\end{figure}

\begin{figure}
    \centering

    \includegraphics[width=0.85\textwidth]{RM2-ALM_verification}

    \caption{Temporal (left) and spatial (right) grid resolution sensitivity
        results for the RM2 ALM RANS model.}

    \label{fig:RM2-ALM-verification}
\end{figure}


\subsection{UNH-RVAT RANS}

Power and drag coefficient curves are plotted for the UNH-RVAT in
Figure~\ref{fig:RVAT-ALM-perf-curves}. The ALM was successful at predicting the
performance tip speed ratios up to $\lambda_0$, which suggests that dynamic
stall was being modeled accurately, but $C_P$ was overpredicted at high
$\lambda$. This may have been caused by the omission of additional parasitic
drag sources such as roughness from exposed bolt heads located far enough from
the axis to have a large effect at high rotation rates, or an underestimation of
the blade--strut connection corner drag coefficient. In Chapter~\ref{chap:RM2}
we showed how these losses can be significant even with carefully smoothed
struts and strut-blade connections. Overprediction of performance at high tip
speed ratio could also be a consequence of the Leishman--Beddoes dynamic stall
model, which can also be seen in the Darrieus VAWT momentum model results shown
in Figure 6.70 of \cite{Para2002}.

\begin{figure}
    \centering

    \includegraphics[width=0.85\textwidth]{RVAT-ALM_perf-curves}

    \caption{Power and drag coefficient curves computed for the UNH-RVAT using
        the actuator line model with RANS.}

    \label{fig:RVAT-ALM-perf-curves}
\end{figure}

Figure~\ref{fig:RVAT-ALM-meancontquiv} shows mean velocity field for the
UNH-RVAT computed by the ALM RANS model. The asymmetry was captured well, along
with some of the vertical flow due to blade tip vortex shedding, though the flow
structure is missing the detail present in the experiments and blade-resolved
RANS simulations. Overall, the wake appears to be over-diffused, which could be
a consequence of the relatively coarse mesh. Note that with the DS and flow
curvature corrections turned off, the direction of the mean swirling motion
reverses, which highlights the importance of resolving the correct azimuthal
location of blade loading.

\begin{figure}
    \centering

    \includegraphics[width=0.9\textwidth]{RVAT-ALM_meancontquiv}

    \caption{Mean velocity field at $x/D=1$ with the LB-SGC model, with and
        without the Goude flow curvature correction.}

    \label{fig:RVAT-ALM-meancontquiv}
\end{figure}

Turbulence kinetic energy contours (including resolved and modeled energy) are
shown in Figure~\ref{fig:RVAT-ALM-kcont}. The ALM was able to resolve the
concentrated area of $k$ on the $+y$ side of the turbine, but the turbulence
generated by the dynamic stall vortex shedding process is absent. This makes
sense since in the ALM, the DS model only modulates the body force term in the
momentum equation, which does not provide a mechanism for mimicking shed
vortices or turbulence.

\begin{figure}
    \centering

    \includegraphics[width=0.85\textwidth]{RVAT-ALM_kcont}

    \caption{Turbulence kinetic energy contours at $x/D=1$ predicted by the
        ALM.}

    \label{fig:RVAT-ALM-kcont}
\end{figure}

Profiles of mean streamwise velocity and turbulence kinetic energy are shown in
Figure~\ref{fig:RVAT-ALM-profiles}. Here the over-diffused or over-recovered
characteristic of the mean velocity deficit seen in
Figure~\ref{fig:RVAT-ALM-meancontquiv} is more apparent. This effect is also
seen in the profile of $k$, where energy is smeared over the center region of
the rotor.

\begin{figure}
    \centering

    \includegraphics[width=0.85\textwidth]{RVAT-ALM_wake-profiles}

    \caption{Mean streamwise velocity (left) and turbulence kinetic energy
        (right) profiles at $z/H=0$ for the UNH-RVAT ALM.}

    \label{fig:RVAT-ALM-profiles}
\end{figure}

Weighted averages for the momentum recovery terms were computed identically as
they were in Chapter~\ref{chap:CFD}, and are plotted in
Figure~\ref{fig:RVAT-ALM-recovery} along with the 3-D blade-resolved RANS
results and experiments. The most glaring discrepancy is the ALM's prediction of
positive cross-stream advection, which is caused by the lack of detail in the
tip vortex shedding. The total for vertical advection, however, is close to that
predicted by the 3-D blade-resolved Spalart--Allmaras model. Levels of turbulent
transport due to eddy viscosity and deceleration due to the adverse pressure
gradient are between those predicted by the 3-D blade-resolved $k$--$\omega$ SST
and SA models. Overall, however, one might expect the total wake recovery rate
to be comparable between all models, which suggests the ALM would be an
effective tool for assessing downstream spacing of subsequent turbines, though
any blade--vortex interaction of very tightly spaced rotors would likely not be
captured.

\begin{figure}
    \centering

    \includegraphics[width=0.85\textwidth]{RVAT-ALM_recovery-bar-chart}

    \caption{Weighted average momentum recovery terms for the UNH-RVAT actuator
        line model with a $k$--$\epsilon$ RANS closure, the two 3-D blade resolved
        RANS models described in Chapter~\ref{chap:CFD}, and the experiments in
        Chapter~\ref{chap:RVAT-baseline}}.

    \label{fig:RVAT-ALM-recovery}
\end{figure}


\subsection{RM2 RANS}

Figure~\ref{fig:RM2-ALM-perf-curves} shows the performance curves computed for
the RM2 by the ALM, and those from the tow tank experiments. As with the high
solidity RVAT, $C_P$ is overpredicted at high $\lambda$. However, $\lambda_0$,
the tip speed ratio of peak power coefficient, is also shifted to the right.
This is indicative of inaccurate dynamic stall modeling, which could possibly be
attributed to one of the models' tuning constants, e.g., the time constant for
the lagged angle of attack. Limited ad hoc testing revealed that $C_P$ at
$\lambda_0$ was more accurately predicted with a lagged angle of attack time
constant roughly double the default value.

\begin{figure}
    \centering

    \includegraphics[width=0.85\textwidth]{RM2-ALM_perf-curves}

    \caption{Power and drag coefficient curves computed for the RM2 using the
        ALM.}

    \label{fig:RM2-ALM-perf-curves}
\end{figure}

Figure~\ref{fig:RM2-ALM-meancontquiv} shows the mean velocity field at 1 m
downstream or $x/D=0.93$ computed by the ALM for the RM2. The mean near-wake
structure looks similar to the RVAT ALM RANS case, but for the RM2, the lack of
detail from blade tip vortex shedding matches more closely with experiments.
However, the vertical flow towards the $x$--$y$ center plane was captured, which
is an important qualitative feature of both CFT near-wakes.

\begin{figure}
    \centering

    \includegraphics[width=0.9\textwidth]{RM2-ALM_meancontquiv}

    \caption{Mean velocity field at $x/D=0.93$ for the RM2 predicted by the
        ALM.}

    \label{fig:RM2-ALM-meancontquiv}
\end{figure}

Figure~\ref{fig:RM2-ALM-kcont} shows the ALM's turbulence kinetic energy
predictions in the near-wake of the RM2. Like for the UNH-RVAT, $k$ appears to
be concentrated on the $+y$ side of the rotor. However, overall levels of
turbulence are lower than for the UNH-RVAT, which is consistent with the
experimental results. However, turbulence generated by the RM2's blade tip
vortex shedding was not captured by the ALM.

\begin{figure}
    \centering

    \includegraphics[width=0.9\textwidth]{RM2-ALM_kcont}

    \caption{Turbulence kinetic energy contours at $x/D=0.93$ behind the RM2
        predicted by the ALM.}

    \label{fig:RM2-ALM-kcont}
\end{figure}

Wake profiles at turbine mid-height in the $x$--$y$ center plane are shown in
Figure~\ref{fig:RM2-ALM-profiles}. Like the UNH-RVAT case, the mean velocity
deficit appears to be recovering too quickly, which may similarly be due to
coarseness of the grid or overprediction of the eddy viscosity. The peak in
turbulence kinetic energy on the $-y$ side of the turbine was also
underpredicted, though was much lower in magnitude compared to the UNH-RVAT,
even in the experimental measurements.

\begin{figure}
    \centering

    \includegraphics[width=0.85\textwidth]{RM2-ALM_wake-profiles}

    \caption{Mean streamwise velocity (left) and turbulence kinetic energy
        (right) profiles at $x/D=0.93$ and $z/H=0$ for the RM2 ALM.}

    \label{fig:RM2-ALM-profiles}
\end{figure}

Finally, a similar mean streamwise momentum transport analysis was undertaken by
computing weighted sums of each term across the entire domain in the $y$--$z$
directions. Similar results as for the UNH-RVAT were obtained, i.e.,
cross-stream advection was predicted to be positive where it should have been
negative, vertical advection was predicted reasonably well, and turbulent
transport due to the eddy viscosity was also relatively large.

\begin{figure}
    \centering

    \includegraphics[width=0.85\textwidth]{RM2-ALM_recovery-bar-chart}

    \caption{Weighted average momentum recovery terms at $x/D=0.93$ for the RM2
        actuator line model with a $k$--$\epsilon$ RANS closure and the experiments
        described in Chapter~\ref{chap:RM2}}.

    \label{fig:RM2-ALM-recovery}
\end{figure}


\subsection{UNH-RVAT LES}

The state-of-the-art in high fidelity turbine array modeling uses the actuator
line method inside large eddy simulation, which allows more of the turbulent
energy spectrum to be directly resolved, only requiring the dynamics of the
smallest scales to be computed by the so-called subgrid-scale (SGS) model. Since
the ALM LES approach has only been reported in the literature for a very low
Reynolds number 2-D CFT \cite{Shamsoddin2014}, and CFTs may provide unique
opportunities to array optimization, which could be explored with LES, it was of
interest to determine how well the ALM coupled with LES might predict wake
dynamics of a higher $Re$ 3-D CFT rotor.

Thus, the UNH-RVAT baseline case was simulated using the Smagorinsky LES
turbulence model \cite{Smagorinsky1963}, which was the first of its kind, and
serves as a good standard for LES modeling since its behavior is well-reported
in the literature. Default Smagorinsky model coefficients were used, LES filter
width was set as the cube root of the local cell volume. The tip speed ratio was
set to oscillate sinusoidally about $\lambda_0$ with a 0.19 magnitude and the
angle of the first peak at 1.4 radians---similar to the rotation presribed in
the blade-resolved RANS simulations discussed in Chapter~\ref{chap:CFD}.

Since the computational cost of LES is significantly higher than RANS,
verification with respect to grid dependence was not performed. Instead, mesh
resolution was chosen relative to similar studies of turbine wake ALM LES. Of
the studies surveyed
\cite{Shamsoddin2014,Archer2013,Martinez-Tossas2015a,Troldborg2007}, the mesh
resolution ranged from 18--64 points per turbine diameter. The mesh here was set
accordingly by using a 16 point per meter base mesh, and refining twice in a
region containing the turbine to produce a 64 point per turbine diameter/height
resolution. The solver was run with a 0.002 second time step, which is
significantly within the limit described by \cite{Martinez-Tossas2015}, where an
actuator line element may not pass through more than one cell per time step.
With these resolutions computation times were $O(10)$ CPU hours per second of
simulated time, which is approximately two orders of magnitude lower than for
blade-resolved RANS.

Mean power coefficient predictions for the UNH-RVAT at its optimal mean tip
speed ratio dropped to 0.20 using the ALM within the large eddy simulation.
However, the amount of information regarding the wake dynamics was greatly
increased, even beyond that of the 3-D blade-resolved RANS.
Figure~\ref{RVAT-ALM-LES-vorticity} shows an instantaneous snapshot of
isosurfaces of vorticity produced by the actuator lines.

\begin{figure}
    \centering

    \includegraphics[width=\textwidth]{RVAT-ALM-LES_vorticity-snapshot}

    \caption{Snapshot of vorticity isosurfaces (colored by their streamwise
        component) at $t=6$ s for the UNH-RVAT LES case.}

    \label{RVAT-ALM-LES-vorticity}
\end{figure}

The near-wake's mean velocity field at $x/D=1$ is shown in
Figure~\ref{RVAT-ALM-LES-vorticity}. Compared with the RANS ALM results, the LES
looks much more like the blade-resolved and experimental results, showing the
clockwise and counterclockwise mean swirling motion on the $-y$ and $+y$ sides
of the rotor, respectively.

\begin{figure}
    \centering

    \includegraphics[width=0.9\textwidth]{RVAT-ALM-LES_meancontquiv}

    \caption{Mean velocity field in the UNH-RVAT near-wake at $x/D=1$ computed
        with the Smagorinsky LES model.}

    \label{fig:RVAT-ALM-LES-meancontquiv}
\end{figure}

Contours turbulence kinetic energy sampled at $x/D=1$ from the large eddy
simulation are plotted in Figure~\ref{fig:RVAT-ALM-LES-kcont}. Compared with
RANS, LES is more able to predict the turbulence generated by the blade tip
vortex shedding and dynamic stall effects, though the overall level of
unsteadiness generated was much lower, especially on the $+y$ side of the rotor.
This is likely a consequence of the SGS modeling, where the vortical structures
generated by the blades remain stable further downstream. Similar effects were
seen in \cite{Martinez-Tossas2015a, Shamsoddin2014}, where higher levels of the
Smagorinsky coefficient delayed vortex breakdown and subsequent higher levels of
turbulence. Figure~\ref{RVAT-ALM-LES-vorticity} shows evidence of these effects,
where the blade bound and tip vortices are still relatively coherent at $x/D=1$.

\begin{figure}
    \centering

    \includegraphics[width=0.85\textwidth]{RVAT-ALM-LES_kcont}

    \caption{Turbulence kinetic energy in the UNH-RVAT near-wake at $x/D=1$
        computed with the Smagorinsky LES model.}

    \label{fig:RVAT-ALM-LES-kcont}
\end{figure}

Mean velocity profiles at the turbine center plane, plotted in
Figure~\ref{fig:RVAT-ALM-LES-profiles}, were predicted more accurately using LES
versus RANS, and rival those of the 3-D blade-resolved models. However,
turbulence kinetic energy profiles did not match as closely with experiments.
Though the qualitative shape was resolved better than that by the RANS ALM
simulation, notably the asymmetric peaks around $y/R = \pm 1$, the turbulence
generated in the large eddy simulation was approximately an order of magnitude
too low. Once again this was probably a result of the subgrid-scale modeling and
its effect on the stability of vortex structures.

\begin{figure}
    \centering

    \includegraphics[width=0.85\textwidth]{RVAT-ALM-LES_wake-profiles}

    \caption{Mean velocity profiles in the UNH-RVAT near-wake at $x/D=1$ and
        $z/H=0$ computed with the Smagorinsky LES model.}

    \label{fig:RVAT-ALM-LES-profiles}
\end{figure}

The planar weighted sums of streamwise momentum recovery terms were computed in
the same way as for the RANS cases with the exception of the turbulent transport
term, which for the LES was computed from the $x$-components of the divergence
of the resolved and SGS Reynolds stress tensors:
\begin{equation}
    \text{Turb. trans.} = - \left( \frac{\partial}{\partial x_j}
    \overline{u^\prime_x u^\prime_j}
    + \frac{\partial}{\partial x_j} R_{xj}
    \right),
\end{equation}
where $u$ indicates the resolved or filtered velocity, and $R$ is the
subgrid-scale Reynolds stress.

Transport term weighted sums computed from the LES results are shown in
Figure~\ref{fig:RVAT-ALM-LES-recovery}. Unlike the RANS ALM cases, the
cross-stream advection contributions are negative, as they are in the experiment
and blade-resolved CFD models. The vertical advection term is positive as
expected, though smaller than in other cases. Interestingly, the turbulent
transport is negative in the LES, meaning the combined effects of the resolved
and SGS stressed are transferring momentum out of the wake. These discrepancies
highlight the difficulty of predicting the near-wake dynamics, the importance of
the SGS model in LES, and the need for data further downstream to test and
refine predictions for wake evolution. For example, setting the Smagorinsky
coefficient higher may induce vortex breakdown earlier, which would raise the
turbulence levels significantly.

\begin{figure}
    \centering

    \includegraphics[width=0.85\textwidth]{RVAT-ALM-LES_recovery-bar-chart}

    \caption{Weighted average momentum recovery terms at $x/D=1$ for the RVAT
        ALM LES using the Smagorinsky SGS model}.

    \label{fig:RVAT-ALM-LES-recovery}
\end{figure}


\section{Conclusions}

An actuator line model for cross-flow turbines, including a Leishman--Beddoes
type dynamic stall model, flow curvature, added mass, and lifting-line based end
effects corrections, was developed and validated against experimental datasets
acquired for high and medium-low solidity rotors at scales where the performance
and near-wake dynamics were essentially Reynolds number independent. When
coupled to a $k$--$\epsilon$ RANS solver ALM simulations took $O(0.1)$ CPU hours
per second of simulated time, while when coupled to a Smagorinsky LES model the
computing time was $O(10)$ hours per second, which represent a four and two
order of magnitude decrease in computational expense versus 3-D blade-resolved
RANS, respectively.

The RANS ALM predicted the UNH-RVAT performance well at tip speed ratios up to
and including that of max power coefficient. The RM2 power coefficient on the
other hand was underpredicted at lower $\lambda$. Both models overestimated
$C_P$ at the highest tip speed ratios, which has been observed in other
simulations using Leishman--Beddoes type dynamic stall models. Possible
explanations include underestimation of added mass effects or blade--strut
connection corner drag, incorrect time constants in the LB DS model, and/or
inaccuracy due to the virtual camber effect. In the present flow curvature
model, the angle of attack is corrected, but the foil coefficient data is not
transformed, meaning the LB DS separation point curve fit parameters are equal
for both positive and negative angles of attack. A foil data transformation
algorithm based on virtual camber should be investigated for future improvement
of the ALM.

The RANS ALM cases were able to match some important qualitative near-wake flow
features, e.g., the mean vertical advection velocity towards the mid-rotor
plane. However, the mean flow structure and turbulence generation due to blade
tip and dynamic stall vortex shedding shows some discrepancy with experimental
and blade-resolved CFD. Extensions to the ALM to deal with these shortcomings
should be developed, e.g., a turbulence injection model as employed by James et
al. \cite{James2010} or a model that will ``turn'' the ALM body force vectors to
approach the effects of leading and trailing edge vortex shedding during dynamic
stall.

The UNH-RVAT was simulated with the ALM embedded within a typical Smagorinsky
LES, which thanks to its lower diffusion and/or dissipation was able to more
accurately capture the large scale vortical flow structures shed by the rotor
blades. Turbulence generated by the blade tip vortex shedding and dynamic stall
region of the blade path was better resolved, but overall lower levels of
turbulence were predicted, which is likely a consequence of the subgrid-scale
model's influence on the stability of shed vortices. This effect was also
apparent in the negative predictions of turbulent transport on the streamwise
momentum recovery. Therefore, subgrid-scale modeling should be investigated
further before applying the ALM LES to array analyses.

The ALM provides a more physical flow description compared to momentum and
vortex models, at a reasonable cost. The ALM has an important advantage over
vortex methods in that the effort solving the flow field remains reasonably
constant as a simulation progresses, whereas free-wake vortex methods typically
increase in computational complexity each time step. This is especially
important for array simulations, where for a given domain adding many turbines
will be cheap in the ALM compared to the rapidly increasing number of vortex
elements generated by each turbine. The ALM also drastically reduces
computational effort compared to blade-resolved CFD, while maintaining the
unsteadiness of the wake not resolved by a conventional actuator disk.
Furthermore, its implementation within finite volume CFD provides a convenient
way to study turbine array siting, since effects like terrain and buildings can
be included, as can no-slip boundary conditions and boundary layer inflow
profiles. Since the \textit{turbinesFoam} library developed here uses OpenFOAM's
modular \texttt{fvOptions} framework, the ALM body force term can be added as-is
to compressible or multiphase solvers, for exploring more complicated physics.
These features would be difficult or impossible to implement in a momentum or
vortex models.

Ultimately, the ALM retains some of the disadvantages of other blade element
methods, but is a the next logical step bridging the gap between low- and
high-fidelity flow modeling, and has been shown to be a method worthy of further
advancement. The prospect of using ALM simulations to drive down computational
cost of RANS by roughly four orders of magnitude---enabling 3-D unsteady
Navier--Stokes simulations to be performed on a typical PC---itself justifies
further development.

At contemporary levels of computing power, the ALM is a useful tool for studying
individual turbines when HPC is not feasible---requiring similar resources as
2-D blade-resolved RANS, but with improved performance predictions. Furthermore,
if arrays of CFTs advance to commercial scale, the ALM combined with LES
represents one of the highest fidelity tools available, and will improve as
turbulence modeling improves. It is therefore expected that this tool will prove
valuable for both the engineering and research of wind and MHK turbines.


\section{Future work}

The actuator line model shows great potential for use in cross-flow turbine
engineering, and this study has inspired many avenues for its improvement.

Firstly, since the fundamental premise of the ALM relies on accurate static foil
coefficient data, it is crucial that more be measured and published openly, to
allow the exploration of new rotor designs with various profiles. There is some
doubt regarding the veracity of the Sheldahl and Klimas NACA foil data
\cite{Bedon2014}. However, competing datasets at large laboratory scale Reynolds
numbers are rare, even for the standard symmetrical foils. CFD may play a role
in generating new data, but models must be rigorously validated, given the
difficulty in predicting stall and its importance on CFT blade loading.

Profile coefficient databases could be expanded to include more dimensions. For
example, since the ALM will almost always involve turbulence modeling, the
effects of local turbulence levels on foil coefficient data could be included,
since stall delay could affect loading significantly, especially for a
cross-flow turbine.

Dynamic stall model empirical constants should be further investigated for their
optimal values within the cross-flow turbine context. Though the
Leishman--Beddoes type models are formulated in terms of a nondimensional time,
there may be a dependence of the various time constants on the pitch rate, as
indicated by the improvement in predictions for the RM2 at optimal tip speed
ratio when using a higher angle of attack lag time constant. Machine learning
techniques could be employed to ``train'' the constants (and/or their
$\lambda$-dependence) to one turbine's power curve, and they could be validated
against the other turbine's data.

Flow curvature corrections should be examined in more detail, and it may be
beneficial to develop models to transform entire foil datasets to match their
virtual camber, as suggested by Migliore et al.~\cite{Migliore1980}. It
also may be advantageous to implement a complementary model to adjust the
direction of the resultant force vector to more accurately generate the blade's
shed vorticity.

Rotor angular speed control via generator models could be included to test the
effectiveness of various control schemes, e.g., sinusoidal $\lambda$ set points,
which have been shown in some cases to improve mean power coefficient
\cite{Strom2015}. Free stream velocity sampling could be implemented to
determine the optimal location for sensing reference velocity used to compute
tip speed ratio and power coefficient, which can become difficult in sheared or
otherwise nonuniform flows \cite{Forbush2015}.

Since \textit{turbinesFoam} is compatible with OpenFOAM's volume of fluid (VOF)
multiphase flow solver \textit{interFoam}, the ALM's effectiveness should be
assessed for predicting the generation and interaction of turbines with surface
waves. The ADV positioning system developed in Chapter~\ref{chap:exp-setup}
could be retrofitted with a wave staff to measure and validate the free surface
disturbance, which could be key to predicting the effects of placing turbines in
channels at high blockage ratio.

Possibly most importantly, the ALM should be validated with far-wake velocity
measurements and/or performance data from two CFTs operating in proximity of
each other. Next, the ALM should be evaluated for its use in modeling CFT arrays
using both RANS and LES. If either can successfully postdict experimentally
measured performance of closely-spaced physical model CFTs, they should then be
coupled with optimization algorithms to automatically explore the design space,
and ultimately drive down the cost of energy.
