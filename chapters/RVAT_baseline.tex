\chapter{Baseline experimental characterization of a high solidity cross-flow
turbine}

For the higher solidity turbine---the UNH-RVAT---the performance and near-wake
were measured in an initial experiment. The Reynolds number dependence,
described in Chapter~\ref{chap:Re-dep}, was observed in a separate experiment.

In the present study the CFT's near-wake was examined in more detail, providing
insight into the mechanisms that improve CFT wake recovery rates compared with
AFTs or other axisymmetric turbulent wakes, with the ultimate goal that these
mechanisms can be replicated in simpler models for use in simulations of large
turbine arrays, where resolving actual turbine geometry is prohibitively
expensive. The ability of one such model---an actuator disk inside a
Reynolds-Averaged Navier--Stokes (RANS) simulation---to predict those defining
characteristics is also assessed.

The dominant scales within the turbine's near-wake are evaluated for their
relative importance, loosely following the conceptual framework presented in
Chamorro et al. \cite{Chamorro2012b}, where the turbine was treated as an
``active filter.'' However, by extending this concept it should be cautioned
that this filter could also be nonlinear, i.e., the spectral modifications of
the inflow are dependent on the spectral distribution itself, not a
superposition of effects at each individual scale. It is also expected that a
CFT will introduce even stronger large (turbine) scale variance into the flow,
due to its cyclical forcing from oscillatory blade angles of attack and relative
velocity. The experiments presented here were performed in a towing tank,
providing a very low turbulence intensity inflow (at least as low as the
instrumentation noise floor), which provides an excellent baseline case for
spectral content added to the flow by the turbine, without any modulation of a
turbulent inflow spectra.

Previous detailed experimental studies with CFTs were generally limited in terms
of Reynolds number due to small geometric scale. On the other hand, as expected,
large-scale measurements were typically performed with lower resolution
instrumentation, and with less control of inflow conditions
\cite{Vermeulen1979}. Brochier et al.~\cite{Brochier1986} employed laser-Doppler
velocimetry (LDV) to acquire detailed flow measurements of a small-scale,
quasi-2-D CFT in dynamic stall. Their study was similar in scope to the work
presented here, but was conducted at very low Reynolds number---approximately
two orders of magnitude smaller than the study presented in this paper. More
recently, Tescione et al. \cite{Tescione2014} performed a detailed experimental
campaign, using particle image velocimetry to illuminate vortex structures in
the wake, and how these interact with each other. However, the question still
remains as to why the CFT wake would recover more quickly than that of an AFT.
The study reported here also examined the three-dimensionality of the wake, as
turbine ``end effects'' will no doubt affect interaction with the free stream.

To summarize, the goals of this study were:
\begin{enumerate}
    
    \item To identify the essential features of the near-wake of a cross-flow
    turbine from experimental measurements acquired at sufficiently high Reynolds
    number.
    
    \item To assess the relative importance of mean and turbulent dynamics on the
    transport of momentum and kinetic energy in the wake.
    
    \item To compare the measured CFT wake to numerical predictions from a uniform
    actuator disk force parameterization implemented inside a RANS model, and
    identify the simulations shortcomings.
    
\end{enumerate}


\section{Experimental test plan}

All experiments were performed at a tow speed of 1 m/s, resulting in a Reynolds
number based on turbine diameter of $Re_D = U_\infty D /\nu = 1 \times 10^6$, or
an approximate blade chord Reynolds number of $Re_c \approx \lambda U_\infty
c/\nu = 2.7 \times 10^5$ for $\lambda=1.9$, where tip speed ratio $\lambda
\equiv \omega D / (2 U_\infty)$. Note that this Reynolds number is high enough
to be considered operating in a $Re$-independent regime \cite{Bravo2007}
\cite{Bachant2014}, which is important considering the relevance of the data to
characterize full-scale behavior, and subsequent validation of simplified
numerical models.


\section{Results and discussion}

\subsection{Data processing}

Data from each tow were extracted where the quantities of interest---torque,
drag, and velocity---had reached an approximately stationary mean value. The
time series were then trimmed further such that they correspond to an integer
number of turbine blade passages, to minimize bias from periodicity. Wake data
collection runs included 30 blade passages, which corresponds to approximately
16.5 seconds or 3300 velocity samples at each measuring station. Drag from the
mounting structure, a.k.a. tare drag, was measured by towing with the turbine
removed, and then subtracted from the turbine measurements to provide a better
estimate of the overall drag on the turbine rotor and shaft alone. Similarly,
tare torque was measured by driving the turbine support shaft and bearings in
air and regressing these values linearly with respect to shaft angular velocity.
This tare torque was then added to the measured turbine torque in
post-processing to provide a more accurate estimate for the true hydrodynamic
torque.

The tank was seeded with 11 $\mu$m mean diameter hollow glass spheres to achieve
adequate beam correlation and signal-to-noise ratios, and no filtering was used
on the velocity data, i.e., statistics were computed using all raw samples from
the measurement interval. For computing partial derivatives of velocity
quantities, a central difference scheme was employed. For the boundaries of the
measurement plane, an inward-facing second-order scheme was used. The data
processing and plotting code, along with the reduced dataset are available from
\cite{Bachant2014_data}.

\subsection{Turbine performance}

Performance curves showing the overall power and drag coefficients of the
turbine are shown in Figure~\ref{fig-perf}. The drag coefficient monotonically
increases with increasing tip speed ratio over the entire range tested, and the
power coefficient reaches a maximum of 26\% around a tip speed ratio
$\lambda=1.8$--$1.9$, where the drag coefficient is 0.96.  These curves informed
the selection of the tip speed ratio $\lambda = 1.9$ as the operating point for
detailed near-wake characterization. We expect that at this tip speed ratio the
turbine blades will be operating in dynamic stall over a part of the turbine
rotation \cite{Scheurich2011}, reaching maximum angles of attack of
approximately 35 degrees, and that this will be a significant contributor to the
near-wake structure. Note that the maximum power coefficient could likely be
improved with simple geometric modifications, e.g., changing the blade pitch
\cite{Fiedler2009},  but for this study the geometry was meant to be as simple
as possible,  therefore the blade pitch was left at zero.

\begin{figure}
    \centering
%    \includegraphics[width=1.0\textwidth]{Figures/perf}
    \caption{Mean turbine power (left) and drag (right) coefficients plotted
        versus tip speed ratio.}
    \label{fig-perf}
\end{figure}


\subsection{Wake characteristics}

The turbine wake was characterized at one rotor diameter downstream of the
turbine axis. The flow measurements mapped out the upper half of the turbine
wake over 3 m in the spanwise direction (of 3.66 m tank width; cf.
Figures~\ref{fig-expsetup} and~\ref{fig-coord}) at a tip speed ratio
$\lambda=1.9$, chosen to represent a typical turbine operating set point at
maximum $C_P$.

The near-wake structure is described in terms of its mean velocity, streamwise
vorticity, Reynolds stresses, and turbulence kinetic energy. Dominant time
scales are identified and evaluated for their contribution to the turbulent
spectra. Finally, the processes that lead to replenishment of momentum and
energy in the wake are identified, with the goal of explaining the CFT's
relatively fast wake recovery.

\subsubsection{Momentum and vorticity}

The mean velocity measured in the wake at one turbine diameter downstream is
shown in Figure~\ref{fig-meanvel}. The most obvious characteristics are the
asymmetry and three-dimensionality of the flow field. The peak momentum deficit
is shifted towards the right side of the turbine (looking upstream). We can also
see the effects of blade tip vortex shedding, where flow is moving downward and
to the left, creating strong streamwise mean vorticity near the blade tip at
$y/R=-1$, contours for which are shown in  Figure~\ref{fig-xvorticity}. The
asymmetry may explain observations of counter-rotating turbine pairs helping
speed wake recovery \cite{Dabiri2011}.

\begin{figure}
    \centering
%    \includegraphics[clip, trim=0 0.25in 0 0.95in, width=0.8\textwidth]{Figures/meancomboquiv}
    \caption{Mean velocity at $\lambda=1.9$. Vectors are cross-stream and
        vertical velocities; contours are streamwise velocity. View is looking 
        upstream, with the turbine frontal area indicated by the solid gray lines.}
    \label{fig-meanvel}
\end{figure}

The generation of streamwise vorticity with opposing directions highlights once
again the three-dimensionality of the wake of this turbine, and its difference
from an axisymmetric swirling wake such as that of an axial-flow turbine. The
two regions of counter-rotating vorticity act like an asymmetrical doublet,
propelling fluid downward towards the turbine centerline.

Compared with the rotating cylinder wake measurements of Lam \cite{Lam2009}, we
see a similar asymmetry in the mean streamwise velocity. The wake is less
asymmetrical with respect to the wake centerline for the turbine compared to the
rotating cylinder for the same non-dimensional rotation rate, although some of
these differences may be due to the cylinder experiments' lower Reynolds
numbers. Compared with the end effects of a finite cylinder wake
\cite{Sumner2004}, we do see generation of a counter-rotating vortex pair,
though for a non-rotating cylinder these are symmetric, which is not the case
for the turbine wake, where the vortex pair seems to be tilted. The turbine's
wake is also similar to a finite cylinder wake in a sense that there is a mean
downward velocity directly behind the wake generator. These similarities and
differences give some perspective regarding the use of cylinders for turbine
simulators in physical model studies of arrays, which was considered by
\cite{Pierce2013}.

Compared with an axial-flow turbine wake \cite{Cal2010}, and a rotating actuator
disk model \cite{Wu2011}, axisymmetric streamwise vorticity or swirl is not
observed here, which is one of the reasons for the wakes' differing dynamics.
This can be explained as a consequence of the conservation of angular momentum,
where the AFT imparts a streamwise rotation along its axis due to torque
generation, where the CFT imparts a torque that is perpendicular to the flow.
The AFT mean swirl only propels fluid away from its axis, while the CFT induces
mean flow into its momentum deficit region, albeit in a asymmetric fashion.

\begin{figure}
    \centering
%    \includegraphics[clip, trim=0 0.3in 0 0.3in,width=0.8\textwidth]{Figures/xvorticity}
    \caption{Contours of mean streamwise vorticity. Negative values indicate
        clockwise vorticity.}
    \label{fig-xvorticity}
\end{figure}

Reynolds shear stress contours, shown in Figure~\ref{fig-restress}, quantify the
replenishment of mean momentum by turbulent transport. Regions of high
$\overline{u'v'}$ as well as the observed asymmetry correspond to the regions of
high gradients of streamwise velocity, indicating regions of high turbulence
production. Note that Bachant and Wosnik \cite{Bachant2013} also showed how the
mean velocity and Reynolds stress varied with tip speed ratio.

\begin{figure}
    \centering
%    \includegraphics[clip, trim=0 0.3in 0 0.5in, width=0.8\textwidth]{Figures/uvcont}
%    \includegraphics[clip, trim=0 0.3in 0 0.5in, width=0.8\textwidth]{Figures/uwcont}
    \caption{Contours of Reynolds shear stress.}
    \label{fig-restress}
\end{figure}

To quantify the relative importance of mean and turbulent transport processes,
we rearrange the streamwise Reynolds-averaged momentum equation to isolate terms
contributing to its streamwise derivative. Assuming the flow is stationary in
the mean and incompressible, the equation becomes

\begin{equation}
\begin{split}
\frac{\p U}{\p x}  =  
\frac{1}{U} \bigg{[}
& - V\frac{\p U}{\p y}
- W\frac{\p U}{\p z} \\
& -\frac{1}{\rho}\frac{\p P}{\p x} \\
& - \frac{\p}{\p x} \overline{u'u'}
- \frac{\p}{\p y} \overline{u'v'}
- \frac{\p}{\p z} \overline{u'w'} \\
& + \nu\left(\frac{\p^2 U}{\p x^2}
+ \frac{\p^2 U}{\p y^2}
+ \frac{\p^2 U}{\p z^2} \right)
\bigg{]}.
\end{split}
\label{eq-momentum}
\end{equation}

From the experimental measurements, all of the terms on the right hand side were
computed except for the streamwise pressure gradient and the streamwise ($x$)
derivatives of the Reynolds and viscous stresses; the latter two terms are
expected to be small under a thin shear layer assumption. Totals for all these
terms, normalized by $D/U_\infty$ to assess nondimensional mean momentum
recovery per turbine diameter of downstream distance are plotted in
Figure~\ref{fig-mombargraph}. We see that as expected, viscous diffusion is
almost negligible. The two Reynolds stress transport terms are both significant,
but the vertical advection term dominates, and the horizontal advection term
contributes negatively, corresponding to streamlines diverging around the
turbine. The flow essentially behaves like an inviscid three-dimensional,
non-axisymmetric turbulent wake. If we sum all the terms we obtain a total wake
recovery rate at $x/D=1$ of 12\% per turbine diameter, which would be
interesting to compare with an axial-flow turbine in similar conditions.

\begin{figure}
    \centering
%    \includegraphics[clip, trim=0 0.25in 0 0.2in, width=0.8\textwidth]{Figures/mombargraph}
    \caption{Estimates for the contributions to mean streamwise momentum recovery
        in the streamwise direction, multiplied by two (due to assumed symmetry), 
        averaged over the measurement plane, and
        normalized by the average streamwise velocity, freestream velocity, and turbine 
        diameter.}
    \label{fig-mombargraph}
\end{figure}


\subsubsection{Dominant time scales}

Spectral densities were computed using a fast Fourier transform (FFT) based
algorithm, with a Hanning window applied to the time series. Spectral values
were averaged over 4 adjacent frequencies to reduce noise and increase
confidence intervals. Figure~\ref{fig-fcont} shows the peak
frequency---normalized by the turbine angular frequency---of the cross-stream
velocity energy spectra at each measurement point. We can see how unsteadiness
in the flow is induced primarily at the blade passage frequency, or three times
$f_\mathrm{turbine}$. Note that some higher peak frequencies can also be seen in
this plot, out in regions of low turbulence intensity. This is due to both the
noise floor of the ADV and the wakes of the guy wire supports.

Spectral concentration $\Psi$ is quantified by normalizing the peak value of the
spectral density by the total variance of the signal multiplied by the
spectrum's Fourier frequency interval $\Delta f$, i.e.,
\begin{equation}
\Psi \equiv \frac{S_{\max} \Delta f}{\sigma^2},
\label{eq-spec_cont}
\end{equation}
where $\Delta f \equiv 1/(N \Delta t)$, $N$ is the total number of samples, and
$\Delta t$ is the sampling interval. For a pure harmonic $\Psi = 1$, indicating
all variance contained within a single frequency. This metric provides further
characterization of the turbulence in terms of its range of scales, rather than
the overall variance or turbulence intensity.

A plot of this concentration is also shown in Figure~\ref{fig-fcont}, where it
is evident that energy is more concentrated towards the side of the  turbine
where the blades move upstream, likely seeing lower angles of attack, and
therefore boundary layer separation occurring further towards the trailing edge
of the blade. This concentration is indicative of more coherent motion, or
extraction of energy/momentum from the flow without intense separation, whereas
on the opposite side of the turbine, turbulence generated by the dynamic stall
process is increasing the diffusion of shed vorticity. This effect can be
further illustrated looking at sample spectra for turbine torque coefficient and
cross-stream velocity components at two points on opposite sides of the turbine,
shown in Figure~\ref{fig-multispec}, where the cross-stream velocity at the side
of the turbine inducing massive separation shows a smaller peak in its spectrum
at the blade passage frequency.

\begin{figure}[h]
    \centering
%    \includegraphics[clip, trim=0 0.3in 0 0.5in, width=0.8\textwidth]{Figures/fpeak_v}
%    \includegraphics[clip, trim=0 0.3in 0 0.3in, width=0.8\textwidth]{Figures/fstrength_v}
    \caption{Cross-stream velocity energy spectra peak frequency normalized by
        turbine angular frequency (top) and spectral concentration (bottom). Note that white 
        regions have values higher than the maximum value of the color bar, caused by both 
        noise and wakes of the guy wire supports.}
    \label{fig-fcont}
\end{figure}

\begin{figure}[h]
    \centering
%    \includegraphics[width=0.98\textwidth]{Figures/multispec}
    \caption{Spectral density for (a) torque coefficient, (b) cross-stream velocity at
        $y/R = -1$; $z/H = 0.25$, and (c) cross-stream velocity at 
        $y/R = 1.5$; $z/H = 0.25$. Dashed vertical lines indicate $[1, 3, 6, 9]$
        times the turbine rotational frequency and the shaded gray region indicates the 
        95\% confidence interval for a $\chi^2$ variable with 8 degrees of freedom---twice 
        the number of frequencies over which spectral values were averaged to reduce noise.}
    \label{fig-multispec}
\end{figure}


\subsubsection{Kinetic energy}

Contours of turbulence kinetic energy are shown in Figure~\ref{fig-kcont}. The
turbulence kinetic energy is concentrated near the top and left side of the
turbine, as a result of the blade tip and dynamic stall vortex shedding,
respectively. The lower turbulence kinetic energy on the $+y$ side of the
turbine corresponds with the more concentrated spectral energy of velocity
unsteadiness in this area. This can be interpreted as the lift-induced vorticity
being shed with less separation compared to the $-y$ side of the turbine, where
turbulence is being generated and redistributing energy across a larger
bandwidth. This brings up one potential improvement to actuator line models:
Modulating turbulence fields at the occurrence of dynamic stall.

\begin{figure}
    \centering
    %\includegraphics[clip, trim=0 0.25in 0 0.3in, width=0.8\textwidth]
    %{Figures/meankcont}
%    \includegraphics[clip, trim=0 0.25in 0 0.3in, width=0.8\textwidth]{Figures/kcont}
    \caption{Contours of turbulence kinetic energy,
        normalized by the mean freestream kinetic energy.
        Turbine frontal area is indicated by solid black lines.}
    \label{fig-kcont}
\end{figure}

Like the analysis of the mean streamwise momentum, the mechanisms which play the
most important role in mean kinetic energy recovery as the wake evolves in the
streamwise direction are now examined. The transport equation for the kinetic
energy $K$ associated with the mean flow \cite{TennekesAndLumley}, rearranged to
isolate the streamwise recovery can be written as
\begin{equation}
\begin{split}
\frac{\p K}{\p x}
=
\frac{1}{U}
\bigg{[}
& - \underbrace{V \frac{\p K}{\p y}}_{y\text{-adv.}}
- \underbrace{W \frac{\p K}{\p z}}_{z\text{-adv.}}
% Pressure work:
- \frac{1}{\rho}\frac{\p}{\p x_j} P U_i \delta_{ij}
% Work by viscous forces
+ \frac{\p}{\p x_j} 2 \nu U_i S_{ij} % Not sure if that's capital S...
% Turbulent transport of K
- \underbrace{
    \frac{1}{2}\frac{\p}{\p x_j} \overline{u_i' u_j'} U_i
}_{\text{Turb. trans.}} \\
% Production of k
& + 
\underbrace{
    \overline{u_i' u_j'} \frac{\p U_i}{\p x_j}
}_{k\text{-prod.}}
% Mean dissipation? Bar could be removed, or no? -- yes, capital letter, no bar.
- 
\underbrace{
    2 \nu S_{ij}S_{ij}
}_{\text{Mean diss.}}
\bigg{]}.
\label{eq-K_full}
\end{split}
\end{equation}
The terms of interest are labeled---advection in the cross-stream and vertical
directions, energy transport by turbulent fluctuations, production of turbulence
kinetic energy $k$ through mean shear, and the dissipation due to mean viscous
shear forces. For the tensor terms, streamwise derivatives were ommitted, since
the measurements were limited to a single streamwise distance.
Table~\ref{tab-eqs} lists all components kept for the computation of each
transport term.

\begin{table}
    \centering
    \begin{tabular}{c|c}
        Term in Eq.~\ref{eq-K_full} & Implementation \\ 
        \hline  
        $y$-adv. & $-\frac{V}{U}\frac{\p K}{\p y}$ \\ 
        $z$-adv.  & $-\frac{W}{U}\frac{\p K}{\p z}$ \\ 
        Turb. trans. ($y$) & $-\frac{1}{2U} \big{[} \frac{\p }{\p y}\overline{u'v'}U 
        + \frac{\p }{\p y}\overline{v'v'}V + \frac{\p }{\p y}\overline{w'v'}W \big{]} $\\ 
        Turb. trans. ($z$)  & $-\frac{1}{2U} \big{[} \frac{\p }{\p z}\overline{u'w'}U 
        + \frac{\p }{\p z}\overline{v'w'}V + \frac{\p }{\p z}\overline{w'w'}W \big{]} $\\ 
        $k$-prod.  & $\frac{1}{U} \big{[} \overline{u'v'}\frac{\p U}{\p y}
        + \overline{v'v'}\frac{\p V}{\p y} 
        + \overline{w'v'}\frac{\p W}{\p y} $ \\
        & $ + \overline{u'w'}\frac{\p U}{\p z}
        + \overline{v'w'}\frac{\p V}{\p z}
        + \overline{w'w'}\frac{\p W}{\p z}
        \big{]} $ \\ 
        Mean diss.   & $ - \frac{2 \nu}{U} \big{[}
        \big{(} \frac{\p U}{\p y} \big{)}^2
        + \big{(} \frac{\p U}{\p z} \big{)}^2
        + \big{(} \frac{\p V}{\p y} \big{)}^2 $ \\
        & $
        + \big{(} \frac{\p V}{\p z} \big{)}^2
        + \big{(} \frac{\p W}{\p y} \big{)}^2
        + \big{(} \frac{\p W}{\p z} \big{)}^2
        \big{]} $ \\ 
    \end{tabular} 
    \caption{Terms used to compute contributions to mean kinetic energy recovery.}
    \label{tab-eqs}
\end{table}

Figure~\ref{fig-Kturbtrans} shows estimates of mean kinetic energy transport by
turbulent fluctuations. The structure of this plot indicates that the turbulent
fluctuations transport mean kinetic energy inward towards regions of lower mean
momentum and lower mean kinetic energy. The signs of the terms plotted here can
be understood from Figures 6 and 4. Note that despite lower turbulence kinetic
energy on the $+y$ side of the turbine, magnitudes of the turbulent transport
are similar on both.

\begin{figure}
    \centering
%    \includegraphics[clip, trim=0 0.25in 0 0.3in, width=0.8\textwidth]{Figures/Kturbtrans}
    \caption{Contours of estimated mean kinetic energy transport by turbulent
        fluctuations, where streamwise derivatives are omitted.
        Turbine frontal area is indicated by solid black lines.}
    \label{fig-Kturbtrans}
\end{figure}

Contributions to mean kinetic energy recovery from various mechanisms were
averaged over the  measurement plane using the trapezoidal rule. Figure~\ref
{fig-Kbargraph}  shows the normalized sum of each quantity to show their
relative size. As with the momentum, the cross-stream mean advection contributes
negatively since the flow is accelerating around the turbine due to its pressure
disturbance, where streamlines diverge.

Viscous dissipation due to mean shear is essentially negligible compared to
the other terms, which is to be expected in a high Reynolds number shear flow a 
short distance downstream of the shear flow (wake) generator (CFT).

Production of turbulence kinetic energy acts to reduce mean kinetic energy, as
expected for the mean kinetic energy equation. The turbulent transport terms,
separated by the direction of their divergence, i.e., ``$y$-turb.'' is a sum of
all terms with $\p / \p y$ in them and ``$z$-turb.'' is a sum of all terms with
$\p / \p z$ in them, are about the same order of magnitude, both roughly an
order of magnitude smaller than the vertical advection term, which is the
largest. It should be re-stated that the terms in Figure~\ref{fig-Kbargraph}
were evaluated in the near wake, in a measurement plane at $x/D=1$ shown in
Figure~\ref{fig-coord}.

Compared to the observations of Kinzel et al. \cite{Kinzel2012}, where the
turbulent transport terms were found to be not large enough to replenish turbine
power output, it can be seen here that it is most likely vertical advection that
plays the most important role in enhancing the wake recovery, not the turbulence
quantities. This is likely a consequence of the unique vorticity generation and
interaction from lift production, (dynamic) stall vortices, and blade end
effects.

\begin{figure}
    \centering
%    \includegraphics[clip, trim=0 0.25in 0 0.2in, width=0.8\textwidth]{Figures/Kbargraph}
    \caption{Estimates for the contributions to mean kinetic energy recovery
        in the streamwise direction, multiplied by two (due to assumed symmetry), 
        averaged over the measurement plane, and
        normalized by the average streamwise advection velocity, freestream kinetic 
        energy, and turbine diameter.}
    \label{fig-Kbargraph}
\end{figure}


\subsection{Comparison with an actuator disk}

The actuator disk model, commonly used in modern large-eddy simulations of wind
farms \cite{Stevens2014}, parameterizes turbine forcing on the flow field as a
steady streamwise force applied over the frontal area of the turbine. The model
is attractive due to its relatively simple implementation; it does not require
the meshing of actual turbine geometry, making it computationally feasible to
simulate large turbine arrays. For example, the turbine array being installed by
ORPC in Cobscook Bay, Maine was laid out with a RANS actuator disk model, where
cross-flow turbines were represented inside the mesh over three cells
\cite{Nelson2013}. Furthermore, the actuator disk force coefficients are not
time dependent, allowing for simulation of, e.g., tidal cycles without the need
for small time steps to resolve unsteady turbine forcing.

Here the experimental measurements are compared and contrasted with the
near-wake of an actuator disk model, to illustrate how this model would
represent a cross-flow turbine in an array simulation. Since one of the
potential benefits of cross-flow turbines is to be able to be spaced more
closely in an array installation, the near-wake dynamics will be more important
than those of an axial-flow turbine.

For axial-flow turbines, the actuator disk model can be enhanced with a rotating
axial velocity component, which gives better results than a simple volume force
\cite{Wu2011}, but to our knowledge this technique has not been applied
(adapted) to cross-flow turbines. Nevertheless, in the present study the
ability of a simple streamwise force distributed over a cylindrical volume to
mimic the near-wake of a cross-flow turbine was evaluated. As described in
previous sections, the asymmetry of the turbine's wake indicates that a uniform
streamwise force from the actuator disk will likely not capture the wake
accurately.

The open source CFD package \textit{OpenFOAM} was used to solve the  RANS
equations with the turbine represented by an actuator disk, which is
implemented as a force or sink in the momentum equation, applied over a selected
volume of cells in the mesh.

The cross-section of the domain is the same as that of the tow tank used for the
experimental measurements. The actuator disk is technically not a disk, but a 1
m diameter by 1 m tall cylinder, mimicking the turbine swept area. The domain
extends 2 m upstream and 8 m downstream.  The background mesh consists of 96,
64, and 48 points in the $x$, $y$, and $z$ directions, respectively. The cells
in the volume enclosed by the actuator disk region are refined by a factor of
two, giving a total of  approximately $3 \times 10^5$ hexahedral cells. A
snapshot of the mesh is shown in Figure~\ref{fig-AD_mesh}. Boundary conditions
are set to approximate the tow tank environment, i.e., the velocity at the
bottom and walls is set to the free stream value. The free surface was not
modeled---the domain had a rigid lid with a slip velocity boundary condition---a
reasonable approximation for low Froude numbers. Inputs to \textit{OpenFOAM}'s
\texttt{actuationDiskSource} are given in Table~\ref{tab-AS}, and the case files
for this simulation are available from \cite{Bachant2014_OF-AS}.


\begin{figure}
    \centering
%    \includegraphics[width=0.9\textwidth]{Figures/AD_mesh}
    \caption{Snapshot of the computational mesh for the actuator disk RANS 
        simulation.}
    \label{fig-AD_mesh}
\end{figure}


\begin{table}
    \begin{center}
        \begin{tabular}{r|l}
            \texttt{Cp} & \texttt{0.26} \\ 
            \texttt{Ct} & \texttt{0.96} \\ 
            \texttt{diskArea} & \texttt{1.0} \\ 
            \texttt{upstreamPoint} & \texttt{(-1.0 0 0)} \\ 
        \end{tabular} 
        \caption{Input parameters for the actuator surface using \textit{OpenFOAM}'s
            \texttt{actuationDiskSource}.}
    \end{center}
    \label{tab-AS}
\end{table}

The turbulence is modeled with a standard $k$-$\epsilon$ closure, with
relatively low levels of inlet turbulence kinetic energy and dissipation, $2
\times 10^{-4}$ and $3 \times 10^{-5}$, respectively. These low free stream
values were chosen to approximate the tow tank ambient conditions.

Results for the mean velocity field are presented in
Figure~\ref{fig-AD_contours}, in a manner similar to that of
Figure~\ref{fig-meanvel}, for comparison. Clearly, the vertical mean flow, which
was determined to be an important driver of near-wake dynamics in the
experiments, is absent. In fact, it contributes negatively to streamwise wake
recovery. This means that all momentum and energy transport back into the
deficit in the wake created by the turbine will need to be facilitated by
turbulent transport (and viscous diffusion, to a much lesser degree). Note also
how acceleration due to blockage is much lower compared to the experiments
despite matching the overall drag coefficient.

Figure~\ref{fig-AD_streamwise} shows the downstream evolution of the centerline
streamwise velocity, and the terms that contribute to its streamwise derivative
averaged over various constant-$x$ planes. Note how very close to the turbine,
the streamwise pressure gradient is contributing significantly to the increase
in $U$, despite the fact that the turbine creates a positive pressure gradient
along the centerline. The large pressure-driven increase in streamwise velocity
makes sense considering the fact that the values are averaged over the entire
cross-section of the domain, which includes a large area of flow acceleration
around the turbine, where the streamwise pressure gradient is negative. Just
after the turbine, $-\p P / \p x$ drops off very quickly and then acts to
decrease streamwise momentum slightly as the static pressure recovers moving
downstream.

We can see that the streamwise momentum recovers very slowly, only starting to
recover around $x=7D$, which can be mostly attributed to the low inflow
turbulence levels, chosen to mimic the tow tank environment. This can be further
understood by looking at the transport terms plotted on the right in
Figure~\ref{fig-AD_streamwise}, where we see all terms are quite small when
compared with the experimental results from the turbine. Rather than the
advection terms, which contribute negatively, the turbulent transport---here
modeled using the $k$--$\epsilon$ eddy viscosity $\nu_t$---is driving the
streamwise evolution, despite being very small. One way to increase the wake
recovery in such a model is to have the actuator disk ``inject'' turbulence
quantities to increase the eddy viscosity \cite{James2011, Nelson2013}. However,
this will likely still not be sufficient to predict evolution and interaction in
closely-spaced arrays of CFTs, since neither the significant vertical mean 
velocities in the CFT wake, nor any coherent vortical structures 
due to blade, shaft, or strut forces are captured by the actuator disk model.

It could be argued that the actuator disk model should not be judged this way as
it is well known that it is a poor predictor of near-wake characteristics,
however, these models are common in engineering practice when calculating
performance of turbine arrays, as previously mentioned. The experiments reported
here have shown that the use of these models would likely be a source of large
uncertainty if applied to cross-flow turbine installations.

\begin{figure}
    \centering
%    \includegraphics[clip, trim=0 0.25in 0 0.5in, width=0.75\textwidth]{Figures/meancomboquiv_AD}
    \caption{Mean velocity predictions at $x/D=1$ from the RANS actuator disk 
        numerical model. Vectors are cross-stream and
        vertical velocities; contours are streamwise velocity.}
    \label{fig-AD_contours}
\end{figure}

\begin{figure}
    \centering 
    \includegraphics[width=\textwidth]{Figures/AD_streamwise}
    \caption{Downstream evolution of the centerline streamwise velocity (left) and
        normalized momentum transport terms (right) averaged over $y$--$z$ slices from  
        the actuator disk RANS simulation.}
    \label{fig-AD_streamwise}
\end{figure}

\section{Conclusions}

Detailed measurements were performed in the near-wake of a vertical axis
cross-flow turbine operating at peak power coefficient. The following essential
features were identified:

\begin{enumerate}
    \item Asymmetry and three-dimensionality in the mean velocity field. 
    
    \item Mean streamwise swirling flow, or vorticity produced by blade tip and
    dynamic stall vortex shedding, which propels fluid towards the wake's center
    and makes mean vertical advection the largest contributor to streamwise
    momentum and mean kinetic energy recovery.
    
    \item Asymmetric turbulence generation due to the effects of dynamic stall
    being more pronounced on one side of the turbine.
\end{enumerate}

The most dominant timescale induced into the wake is the blade passage period. 
The $+y$ side of the turbine contains more coherent motion at this
frequency, as stalling is less prevalent there. The reduced separation due to 
stall also leads to lower magnitudes of turbulence kinetic energy on the $+y$ 
side. 

Regarding recovery of the mean streamwise momentum and kinetic energy, it was
calculated from the wake velocity measurements that vertical advection is more
than twice as large as transport by turbulent fluctuations, which may explain
why CFT wakes entrain free stream kinetic energy more effectively than their AFT
counterparts. The importance of the vertical flow created by the turbine showed
that array flow simulations will need to be carried out in three dimensions to
produce accurate results. Considering how high power coefficient estimates are
in 2-D simulations \cite{Li2013}, it logically follows that 3-D effects
significantly decrease the power output of a single turbine, but this uncaptured
power helps pull more power from outside the array. This also raises interesting
questions with respect to how cross-flow turbine blades should be
``terminated'', i.e., reducing blade tip vortices by winglets or inhibiting them
with end struts or end disks, as commonly done, may increase the performance of
individual CFTs, however, free ends that produce tip vortices may be
advantageous in an array setting to increase the wake recovery rate.

A commonly used, simple turbine forcing parameterization---an actuator
disk---was assessed for predicting the near-wake characteristics of this turbine
with a RANS simulation. The actuator disk model was found to be a poor
representation of a CFT, despite its computational efficiency and use in
research and industry. It is suggested that an actuator disk with a non-uniform
distribution of force, and an actuator line, may produce the asymmetry and
unsteadiness that is characteristic of the CFT near-wake and will be necessary
to predict performance of closely-spaced arrays that cross-flow turbines are
thought to enable.