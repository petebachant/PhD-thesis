\chapter{Development of an experimental setup for measuring the performance and
    near-wake of cross-flow turbines at large laboratory
    scale}\label{chap:exp-setup}

In 2011, a turbine test bed was developed for measuring the performance
(mechanical power and overall rotor drag or thrust) of large laboratory scale
($O(1)$ m\textsuperscript{2} frontal area) cross-flow turbines in the University
of New Hampshire (UNH) tow tank \cite{Bachant2011-MS}, a 36.6 m long, 3.66 m
wide, and 2.44 m deep facility, pictured in Figure~\ref{fig:tow-tank}, which was
capable of towing up to approximately 1.4 m/s. The turbine-specific
instrumentation consisted of a mounting frame built from NACA 0020 hydrofoil
struts, a hydraulic disk brake for turbine loading, an Interface T8 200 Nm
capacity rotary torque transducer, and a 54 pulse-per-rev magnetic pickup for
measuring shaft speed. The frame was mounted to the carriage via
streamwise-oriented linear bearings and was held in place by a pair of Sentran
ZB3 500 lbf load cells that allowed measurement of total streamwise rotor drag.

\begin{figure}
    \centering
    
    \includegraphics[width=0.9\textwidth]{tow-tank.png}
    
    \caption{The University of New Hampshire's wave and tow tank, located in the
        Jere A. Chase Ocean Engineering Laboratory.}
    
    \label{fig:tow-tank}
\end{figure}

Despite its usefulness in collecting a relatively small amount of data for two
helical cross-flow turbines~\cite{Bachant2015-RE}, this ``first generation''
system had some issues to be addressed:
\begin{itemize}
    \item No control over turbine shaft angular velocity. This made operation at
    tip speed ratio below peak torque impossible.
    
    \item Fully manual starting and load application. This limited resolution of
    the applied torque, and took considerable effort to perform experiments on
    the order of 100 tows, since a person had to ride on the tow carriage to
    adjust and apply the load torque.
    
    \item Open loop speed and manual position control of the tow carriage. This
    also took considerable effort to operate experiments, since the operator had
    to estimate braking distance to ensure the carriage did not hit the tank
    ends.
    
    \item Low carriage acceleration. The carriage acceleration was on the order
    of 0.1 m/s\textsuperscript{2}, which limited the steady state turbine
    operating duration to a few seconds.
    
    \item Low frequency resonance in the tow member. A long 0.25 inch diameter
    wire rope was used to tow the carriage, which resonated longitudinally with
    the significant variation of streamwise forces from the turbine.
\end{itemize}

In addition to the above issues, in order to meet the data acquisition goals, it
was necessary to measure turbine wake flows. These concerns required major
renovations, upgrades, and additions to the tow tank and turbine test bed
motion, control, and data acquisition systems. Furthermore, it was desirable to
automate the entire system to increase both data quality and quantity. These
changes were made possible thanks to an infrastructure grant from the US
Department of Energy (DOE).


\section{Modifications to the UNH tow tank}

The ``foundation'' of the experimental setup was the tow tank, which was
addressed first. The main goals for the tow tank upgrades were to increase max
speed and acceleration, add closed-loop positioning and velocity control,
stiffen the tow member to reduce longitudinal resonance, and add onboard power
and networking to the carriage for data acquisition and other peripherals. A
summary of the old system and new target specifications is shown in
Table~\ref{tab:tow-tank-specs}.

\begin{table}
\centering
\begin{tabular}{c|c|c}
Spec & Old system & Target \\ 
\hline
Maximum speed & 1.4 m/s  & 3.0 m/s \\ 
Maximum acceleration & 0.1 m/s$^2$ & 2.0 m/s$^2$ \\ 
Control system & Open loop velocity only & Closed loop position/velocity \\ 
Onboard power & $4\times12$ V batteries & Continuous 120 and 220 VAC \\ 
\end{tabular}
\caption{Specifications summary for existing and upgraded tow tank systems.} 
\label{tab:tow-tank-specs}
\end{table}


\subsection{Linear guides}

The previous linear guide system, shown in Figure~\ref{fig:old-linear-guides},
consisted of a ``master'' guide constructed from $4 \times 4$ inch fiberglass
tubing, and a ``slave'' guide constructed from aluminum angle, on which plastic
wheels rode. Over time, the fiberglass tubing had failed structurally and was
covered with stainless steel bars fixed with double-sided tape. These bars
shifted around considerably during towing and were a source of noise in the
measurements.

A new set of linear guides was designed from 1.25 inch diameter Thomson 440C
stainless steel linear shafts and super self-aligning linear bearings, shown in
Figure~\ref{fig:new-linear-guides}. The existing carriage was modified to
retrofit the linear bearings, and a series of parts were designed to adapt the
stainless shafts to the existing quasi-level mounting surfaces, which helped
keep cost down. The shafts were mounted via 3/8-24 inch threaded rods in
oversized holes to allow adjustment in all three dimensions, a concept which was
inspired by similar linear guide setups at the University of Minnesota's Saint
Anthony Falls Laboratory (SAFL). The shafts were aligned in the cross-tank
direction using a piece of monofilament line stretched along the path. The
vertical alignment was set by spacing the shaft from its mounting surface
equally along the path via machined blocks. When the existing level surfaces
were set in 1996, these were measured to be level within $\pm 1/16$
inches~\cite{Darnell1996}.

\begin{figure}
    \centering
    
    \begin{subfigure}{0.47\textwidth}
        \includegraphics[width=\textwidth]{old-tow-tank-linear-guides}
        \caption{Previous linear guide system.}
        \label{fig:old-linear-guides}
    \end{subfigure}
    \begin{subfigure}{0.47\textwidth}
        \includegraphics[width=\textwidth]{new-tow-tank-linear-guides}
        \caption{Upgraded linear guides.}
        \label{fig:new-linear-guides}
    \end{subfigure}
    
    \caption{Linear guide system (on ``master'' side) (a) before and (b) after
        upgrades.}
\end{figure}


\subsection{Motion and control}

The tow tank's previous motion system consisted of a 10 horsepower AC induction
motor powered by a Yaskawa V7 variable frequency drive. The motor was coupled to
a speed reducing gearbox, on which a pulley was mounted to drive a 0.25 inch
diameter wire rope. It was seen in previous testing that this system had very
low acceleration ($\sim 0.1$ m/s$^2$), which severely reduced steady state
towing durations. The relatively low spring constant of the wire rope tow member
also gave the system a low natural frequency, which resonated due to cross-flow
turbines' cyclic forcing. Furthermore, the system was only velocity-controlled,
and in an open-loop manner. This meant positioning was done manually, which took
a skilled operator, and reduced usable tank length further to allow for coasting
to a stop.

These issues were addressed by changing the motor to a 26.1 maximum horsepower
Kollmorgen AKM82 permanent magnet servo motor and 10:1 gearbox, shown installed
in Figure~\ref{fig:tow-servo}, which was sized to tow turbines with 1 m$^2$
frontal area up to 3 m/s, while accelerating at 2 m/s$^2$. The motor was powered
by a Kollmorgen S700 servo drive, controlled by an 8-axis ACS NTM EtherCAT
master controller, providing closed loop position and velocity control. A series
of emergency stop buttons were also installed to increase the safety of the
system.

\begin{figure}
    \centering
    
    \includegraphics[clip, trim=0 0 0 1.5in, width=0.8\textwidth]{tow-servo}
    
    \caption{Upgraded tow system servo motor, gearbox, and custom-designed
        timing belt pulley housing.} 
    
    \label{fig:tow-servo}
\end{figure}

A 7.5 cm wide steel-reinforced polyurethane timing belt was chosen as the new
drive member. The most robust timing belt profile---an ATL20---was chosen for
maximum stiffness per unit width. Custom timing belt and pulley housings were
designed to move both the upper and lower runs of the belt above the tank wall,
shortening the overall length, which when combined with the higher specific
stiffness belt increased the total drive member spring constant roughly by a
factor of 7.


\subsection{Data acquisition and onboard accessories}

The previous generation tow tank data acquisition (DAQ) system was based around
an onboard PC, powered by a set of four 12 V automotive batteries. This was done
to avoid the complexity of running power out to the carriage \cite{Darnell1996}.
The DAQ PC was accessed wirelessly via Windows Remote Desktop to control any DAQ
applications. The PC that sent the control signal to the inverter drive was a
separate machine, which meant users had to work with at least two interfaces to
specify DAQ and motion parameters. This also made it difficult to synchronize
motion with data acquisition, e.g., triggering data collection at a certain
location.

A new DAQ system was designed based around a National Instruments (NI) 9188
CompactDAQ Ethernet chassis. NI 9237, 9205, 9401, and 9411 modules were
installed for analog bridge, analog voltage, digital, and quadrature encoder
signals, respectively. A single CAT5e cable was dedicated for this system.
Additional cables were run for the EtherCAT and Internet connectivity on the
carriage. An 8-port Ethernet-serial server was installed for accessing serial
devices, e.g., the Nortek Vectrino+, described later. For measuring carriage
speed, and therefore inflow velocity, a Renishaw LM15 linear encoder with 10
$\mu$m resolution was installed and connected to the NI 9411 module. Networking,
power, and control signal cables were run through an igus cable carrier,
installed along the ``slave'' or $+y$ side of the tank.

Requirements for onboard power were derived from the goal of fully automating
both motion and data acquisition. It was also determined that the UNH ME
department's high frame rate particle image velocimetry (HFR-PIV) system would
be used on the carriage at some point, which included laser power supplies and a
laser chiller that could not be powered by the previous generation's isolated
battery/inverter system. 

An onboard electronics cabinet was designed by Minarik, Inc. as part of the
upgraded motion system. A 45 amp, 120 VAC circuit and 20 amp, 240 VAC single
phase power cable were run through the cable carrier to power outlets on the
side of the onboard electronics cabinet. An additional 240 VAC three-phase
supply was connected to a Kollmorgen AKD servo drive, also installed in the
cabinet, which was sized to power a servo motor to control turbine shaft
position and speed. The AKD drive's digital outputs were setup for triggering
instrumentation, e.g., the NI 9188 chassis, via the main motion controller.


\section{Upgraded turbine test bed}

For this work, the turbine test bed was kept mostly intact, but modified for
fully-automated operation. To reduce low frequency resonance in the frame caused
by turbine side forces, and help redistribute some of the streamwise force from
turbines towed at higher speeds, two pairs of steel guy wires were added. These
solutions were chosen based on a finite element analysis (FEA) of the turbine
mounting frame, which showed more improvement regarding stiffening in the
desired directions compared to simply adding 45 degree flat bar braces in the
corner joints. To ensure drag from the outer guy wires was included in the
overall streamwise force measurement, an additional set of linear bearings was
added to the carriage for their connection.

Summaries of turbine test bed sensors and instrumentation are presented in
Table~\ref{tab:sensors} and Table~\ref{tab:instrumentation}, respectively. A
drawing and photo of the test bed are shown in
Figure~\ref{fig:turbine-test-bed}. Details on each subsystem are presented in
following sections.

\begin{table}
    \centering
    \begin{tabular}{c|c|c|c}
        Measured quantity & Device type & Mfg. \& model & Nominal accuracy \\
        \hline 
        Carriage position & Linear encoder & Renishaw LM15 & 10 $\mu$m/pulse \cite{RenishawLM15}\\
        Turbine angle & Servo encoder output & Kollmorgen AKD & 10$^5$ pulse/rev \cite{KollmorgenAKD}\\
        Turbine torque & Rotary transducer & Interface T8-200 & $\pm$0.5 Nm \cite{InterfaceT8}\\ 
        Turbine torque (2) & Load cell (\& arm) & Sentran ZB3-200 & $\pm$0.2 Nm \cite{SentranZB}\\
        Drag force, left & Load cell & Sentran ZB3-500 & $\pm$0.6 N \cite{SentranZB}\\
        Drag force, right & Load cell & Sentran ZB3-500 & $\pm$0.6 N \cite{SentranZB}\\
        Fluid velocity & ADV & Nortek Vectrino+ & $\pm$0.5\% $\pm$1 mm/s \cite{NortekVectrino}\\
    \end{tabular}
    
    \caption{Turbine test bed sensor details. Note that ``(2)'' denotes a
        secondary redundant measurement. ``Turbine torque (2)'' nominal accuracy
        estimated by combining load cell accuracy and arm machining tolerances ($\pm
        1 \times 10^{-4}$ m) as root-sum-square.}
    
    \label{tab:sensors}
\end{table}

\begin{table}
    \centering
    \begin{tabular}{c|c|c}
        Measured quantity & Device type & Mfg. \& model \\
        \hline 
        Carriage position & Differential counter & NI 9411 \\
        Carriage velocity (2) & Motion controller & ACS NTM \\
        Turbine angle & Differential counter & NI 9411 \\
        Turbine RPM (2) & Motion controller & ACS NTM \\
        Turbine torque & Analog voltage input & NI 9205 \\ 
        Turbine torque (2) & Analog bridge input & NI 9237 \\
        Drag force, left & Analog bridge input & NI 9237 \\
        Drag force, right & Analog bridge input & NI 9237 \\
    \end{tabular}
    
    \caption{Details of the instrumentation used to perform experiments with the
        turbine test bed. Note that ``(2)'' denotes a secondary redundant
        measurement.} 
    
    \label{tab:instrumentation}
\end{table}

\begin{figure}
    \centering
    \begin{subfigure}[t]{\textwidth}
        \centering
        
        \includegraphics[width=0.75\textwidth]{figures/turbine-test-bed-photo}
        
        \caption{} 
        
        \label{fig:turbine-test-bed-photo}
    \end{subfigure}
    
    \begin{subfigure}[t]{\textwidth}
        \centering
        
        \includegraphics[width=0.85\textwidth]{turbine-test-bed-drawing}
        
        \caption{}
        
        \label{fig:turbine-test-bed-drawing}
    \end{subfigure}
    
    \caption{Turbine test bed photo (a) and drawing (b).}
    
    \label{fig:turbine-test-bed}
\end{figure}


\subsection{Turbine loading, speed control, torque, and drag measurement}

In order to control turbine shaft angular velocity, a Kollmorgen AKM62Q servo
motor and 20:1 ratio gearhead were added with a custom retrofit mounting plate
and housing. Two zero-backlash R+W EKH/300/B curved jaw couplings were added
above and below the rotary torque transducer. An additional torque measurement
system was added by mounting the servo/gearhead assembly to a slewing ring
bearing, and holding its mounting housing in place by a Sentran ZB3 200 lbf load
cell attached at a fixed distance by a 16 inch long arm. This system served as a
redundant torque measurement for values up to 200 Nm, and extended the maximum
torque range to approximately 360 Nm.

Turbine shaft angle was measured via the AKD drive's emulated encoder output,
set to $5 \times 10^3$ (pre-gearbox) or $1 \times 10^5$ (post-gearbox)
lines-per-rev in an A-quad-B configuration. This signal was sampled by either
the NI 9401 (experiments reported in Chapter~\ref{chap:RVAT-baseline}) or the NI
9411 (experiments reported in Chapters~\ref{chap:Re-dep} and \ref{chap:RM2})
modules. Shaft speed was computed by differentiating the angle time series with
a second order central difference scheme. A moving average filter $\sim$10
samples wide was applied to smooth the resulting $\omega$ time series such that
it agreed nearly identically with the shaft RPM measured by the servo motor's
resolver feedback as sampled by the motion controller.

The two drag slide assemblies for overall streamwise drag or thrust measurement
were retained from the previous setup, which is described in
\cite{Bachant2011-MS}. All turbine performance related signals were sampled by
modules in the NI 9188 CompactDAQ chassis at a 2 kHz sample rate.


\subsection{Wake measurement system}

In order to characterize turbine wakes, a Nortek Vectrino+ acoustic Doppler
velocimeter (ADV) was purchased with a Hubbard Fund grant from UNH. An ADV is
capable of measuring three components of velocity at a single point in space
(technically over a small volume), and the Vectrino+ was set to sample at 200
Hz. This system was considered desirable compared with hot wire or hot film
anemometry as it required no calibrations, and the sensor element is
significantly more robust. Spatial resolution is typically lower---the
Vectrino's measurement volume is 6 mm in diameter \cite{NortekVectrino}---but
this is still small compared with the typical length scale of a turbine model.
ADV was also preferable to laser Doppler velocimetry (LDV) in this case since
the tow carriage is a high vibration environment, which would make LDV alignment
a challenge.

A $y$--$z$ axis positioning system was designed for the Vectrino probe. This
system consisted of two Velmex BiSlide linear stages---the $y$-axis driven by
belt and the $z$-axis by ball screw. Both drive systems were powered by stepper
motors with approximately 0.001 inch resolution. These motors were driven by an
ACS UDMlc EtherCAT drive, connected to the tow tank's main motion controller for
integrated synchronous motion.

When operating the Vectrino, the tank was seeded with 11 $\mathrm{\mu}$m mean
diameter hollow glass spheres. Seeding was added along the tank length,
generally at the surface, and was mixed by towing the turbine through the tank.
This process was repeated until the Vectrino's signal-to-noise ratio (SNR) was
approximately above 12 dB. Seeding was added throughout experiments as
necessary---totaling approximately 1--5 cups (dry) per day. Note that while
acquiring ADV data the $y$- and $z$-axis stepper drive had to be disabled to
reduce noise. The axes were re-enabled to position the probe before each run.


\subsection{Software}

Software was developed to automate the entire turbine testing process. Dubbed
\textit{TurbineDAQ}, the desktop application was written in Python due to its
reputation as a good ``glue'' language for systems integration. The graphical
user interface (GUI), shown in Figure~\ref{fig:TurbineDAQ}, was built using the
PyQt bindings to the Qt framework. Communication with the tow tank's motion
controller, data acquisition system, and ADV were integrated into a single
application. This, combined with the ability to load and automatically execute
test matrices in comma-separated value (CSV) format, allowed for experiments
consisting of thousands of tows, where the previous generation test bed could
only realistically achieve around 100. Note that the software was developed
after the experiments in Chapter~\ref{chap:RVAT-baseline}, for which turbine tip
speed ratio, Vectrino positioning, and data collection parameters were input
manually. However, \textit{TurbineDAQ} was developed before and employed for
both the larger experiments described in Chapter~\ref{chap:Re-dep} and
Chapter~\ref{chap:RM2}.

\begin{figure}
    \centering
    
    \includegraphics[width=0.95\textwidth]{TurbineDAQ}
    
    \caption{\textit{TurbineDAQ} turbine test bed experiment automation software
        graphical interface.}
    
    \label{fig:TurbineDAQ}
\end{figure}


\subsection{Tare drag and torque compensation} 

The drag and torque measurement systems were set up in such a way that raw
measurements for drag would include all submerged gear and torque would include
all friction below the transducer along with the turbine shaft torque. To
compensate, tare torque and drag runs were performed to measure the shaft
bearing friction torque and turbine mounting frame drag, respectively. Tare drag
runs were performed for each tow speed in each experiment, for which the mean
value was subtracted in data processing to estimate rotor drag alone. Tare
torque runs were performed by rotating the turbine shaft (without blades) in air
at constant angular velocity for a specified duration, over the range of angular
velocities used throughout the experiment. Tare torque was then fit with a
linear regression versus shaft angular velocity, and added to the measured
turbine torque in post-processing.


\subsection{Synchronization of instrumentation subsystems}

The three data acquisition instrumentation subsystems---motion controller, NI
DAQ (performance measurements), and Vectrino+ (wake velocity
measurements)---were set to begin sampling at precisely the same time each run,
after being triggered by a TTL pulse created by the motion controller. This
strategy retains synchronization for all performance signal samples (tow speed,
torque, drag, angular velocity), ensuring precise calculation of, e.g., power
coefficient. Since there is also synchronization of the initial sample from each
three subsystems, correlation of events in the performance and wake signals is
also possible.


\subsection{Calibrations}

Factory calibrations for all instrumentation were used for the experiments
described in Chapters~\ref{chap:RVAT-baseline} and \ref{chap:Re-dep}. The drag
slide and torque arm assemblies were calibrated out of the tank using the
fixtures shown in Figure~\ref{fig:calibration-fixtures}, in which a Sentran ZB3
500 lbf capacity load cell and indicator were used for input values. The
reference load cell and indicator were calibrated as a full system from the
factory and remained connected to each other at all times. The drag slide and
torque arm fixtures were loaded incrementally using a 3/4-16 inch threaded rod,
nut, and self-lubricating thrust bearing.

\begin{figure}
    \centering
    \begin{subfigure}{0.9\textwidth}
        \includegraphics[clip, trim=0 0 0 3in,
        width=\textwidth]{torque-arm-calibration} \caption{}
    \end{subfigure}
    
    \begin{subfigure}{0.9\textwidth}
        \includegraphics[width=\textwidth]{drag-slide-calibration}
        \caption{}
    \end{subfigure}
    
    \caption{Torque arm (a) and drag slide (b) calibration fixtures. Note that
        the same load cell, indicator, and thrust bearing are used for both setups.}
    
    \label{fig:calibration-fixtures}
\end{figure}

Before the RM2 experiment described later in Chapter~\ref{chap:RM2}, traceable
calibration certificates were obtained for the Interface T8-200 torque
transducer and NI 9205 and NI 9237 modules. In 2014, the drag slide and torque
arm assemblies were recalibrated using the same fixture and Sentran ZB3 load
cell and indicator described above, for which a new traceable calibration
certificate was obtained. Values before and after the recalibration are
presented in Table~\ref{tab:calibrations}.

\begin{table}
    \centering
\begin{tabular}{c|c|c|c}
    Signal & Calibration 1 & Calibration 2 & Difference \\ 
    \hline 
    Torque trans. & 40.0000 Nm/V & 39.8380 Nm/V & -0.4 \% \\ 
    Torque arm & 122531 Nm/V/V & 123437 Nm/V/V & 0.7 \% \\ 
    Drag left & 743104 N/V/V & 742830 N/V/V & -0.1 \% \\ 
    Drag right & 740137 N/V/V & 742400 N/V/V & 0.3 \% \\ 
\end{tabular}
    \caption{Calibration slopes used for experimental measurements. Calibration
        1 was used for the experiments described in Chapter~\ref{chap:RVAT-baseline}
        and Chapter~\ref{chap:Re-dep}. Calibration 2 was used for those in
        Chapter~\ref{chap:RM2}.}
    
    \label{tab:calibrations}
\end{table}


\section{Determining tank settling time}

For each experiment, sample turbine tows were performed at each speed to
determine the amount of time taken between runs such that the tank has settled
adequately, i.e., background turbulence and any large scale mean flows have been
dissipated. This was assessed by towing the turbine, bringing the carriage back
to mid-tank, and allowing the Vectrino to continue recording velocity data,
monitoring the mean and standard deviation of the signals.

Turbulence intensity tended to decay quickly, while the lower frequency
fluctuations were damped with the help of the tank's ``beach,'' designed for
absorbing waves created by the paddle-style wavemaker. Settling times ranged
from 140 seconds for tows at 0.3 m/s up to 360 seconds for those at 1.4 m/s.
These values were then included in the experiment configuration---one value for
each tow speed, to be used by \textit{TurbineDAQ} as wait times between
automated runs.


\section{Experimental uncertainty}\label{sec:uncertainty}

For each set of experimental measurements, uncertainty was estimated from both
systematic and random errors. Random error was computed on a
per-rotor-revolution basis and systematic errors were estimated from the sensor
calibrations or datasheets. Computation and combination of both error sources
and their propagation into derived quantities, followed Coleman and Steele
\cite{ColemanSteele}, which is summarized below.

An expanded uncertainty interval with 95\% confidence
\begin{equation}
    U_{95} = t_{95} u_c,
\end{equation}
was computed for mean power coefficient $C_P$, mean rotor drag coefficient
$C_D$, and mean wake velocities, where $t_{95}$ is the value from the Student
$t$-distribution for a 95\% confidence interval and $u_c$ is the combined
standard uncertainty, which is given by
\begin{equation}
    u_X^2 = s_{\bar{X}}^2 + b_X^2,
\end{equation}
where $s_{\bar{X}} = s_X/\sqrt{N}$ is the sample standard deviation of the mean
per turbine revolution, the total of which is $N$. The systematic uncertainty
$b_X$ is computed as
\begin{equation}
    b_{X}^2 = \sum_{i=1}^J \left( \frac{\partial X}{\partial x_i} \right)^2
    b_{x_i}^2,
\end{equation}
where $x_i$ is a primitive quantity used to calculate $X$ (e.g., $T$, $\omega$,
and $U_\infty$ for calculating $C_P$), and $b_{x_i}$ is the primitive quantity's
systematic uncertainty, estimated as half the value listed on the sensor
manufacturer's documentation. Values for each of the sensors are listed in
Table~\ref{tab:sensors}.

Selecting the Student $t$-statistic $t_{95}$ requires an estimate for the number
of degrees of freedom $\nu_X$, which was obtained using the Welch--Satterthwaite
formula \cite{ColemanSteele}:
\begin{equation}
    \nu_X = \frac{\left(s_X^2 + \sum_{k=1}^M b_k^2 \right)^2} {s_X^4/\nu_{s_X} +
    \sum_{k=1}^M b_k^4/\nu_{b_k}},
\end{equation}
where $\nu_{s_X}$ is the number of degrees of freedom associated with $s_X$ and
$\nu_{b_k}$ is the number of degrees of freedom associated with $b_k$.
$\nu_{s_X}$ is assumed to be $(N-1)$. $\nu_{b_k}$ was estimated as
\begin{equation}
    \nu_{b_k} = \frac{1}{2} \left( \frac{\Delta b_k}{b_k} \right)^{-2},
\end{equation}
where the quantity in parentheses is the relative uncertainty of $b_k$, which
was assumed to be 0.25.


\section{Blockage}

Placing a turbine in a confined environment such as a towing tank will force
flow through the turbine at higher velocity compared with a free case, where
streamlines are allowed to diverge. Consequently, higher levels of blockage will
lead to increased turbine performance, and a shift in optimal operating
parameters, i.e., tip speed ratio. In order to for experiments to be relevant to
others in the literature, blockage effects must either be corrected for, or the
blockage ratio should be kept to a typical value seen in other studies.

Most numerical models have the ability to include finite domains or walls, which
makes uncorrected applicable to validation; applying blockage corrections may
even complicate validation efforts. Furthermore, blockage will be non-zero in
most MHK cases, where turbines are placed in rivers or channels, the effects of
which need to be predicted as well. Since a definitive blockage correction for
CFTs has not yet been established \cite{Cavagnaro2014, Dossena2015}, and a 1
m\textsuperscript{2} turbine creates a reasonably low blockage---on the order of
10\%---no corrections were applied.


\section{Data processing}

All experiments performed with the turbine test bed were analyzed similarly.
For each tow speed, a relevant quasi-steady duration was selected by manually
inspecting a plot of the $C_P$ time series. This interval was then truncated to
include a whole number of blade passages. Relevant statistics were then
calculated over this duration.

To calculate turbine RPM from shaft angle, the encoder signal was differentiated
using a second order central difference scheme, after which an 8 sample wide
moving average smoothing filter was applied to match the noise level present in
the redundant turbine RPM measurement from the motion controller. A similar
approach was used for calculating tow carriage speed $U_\infty$ from carriage
position measurements. Power and drag coefficients were calculated as
instantaneous quantities from the carriage speed as

\begin{equation}
    C_P = \frac{T \omega}{\frac{1}{2} \rho A U_\infty^3}
\end{equation}
and
\begin{equation}
    C_D = \frac{F_\mathrm{drag}}{\frac{1}{2} \rho A U_\infty^2},
\end{equation}
where $\rho$ is the fluid density (assumed to be a nominal 1000
kg/m\textsuperscript{3}) and $A$ is the turbine frontal area $DH$.


\section{Turbine models}

Two physical turbine models were designed and built. The first was intended to
be a geometrically simple high-solidity turbine. This simplicity was achieved
with symmetrical foil profiles, square frontal rotor area, and a rectangular
blade planform. This turbine was constructed from materials donated by Lucid
Energy Technologies, LLP, and dubbed the ``Reference Vertical-Axis Turbine'' or
UNH-RVAT.

The second turbine was designed and built as part of a measurement task for
Sandia National Laboratories (SNL), in collaboration with the US Department of
Energy (DOE), which is described in more detail later in Chapter~\ref{chap:RM2}.
The so-called ``Reference Model 2'' (RM2) was developed by SNL to be a standard
cross-flow turbine for which modelers could validate their predictions
\cite{Neary2014}. The RM2 was designed using Sandia's CACTUS vortex line code
\cite{Barone2011}, and its low-to-medium solidity made it a nice complement to
the UNH-RVAT for testing the robustness of numerical models to varying solidity.


\subsection{UNH-RVAT}

The UNH-RVAT turbine was constructed from straight 14 cm chord length NACA 0020
extrusions, used for both the blades and struts. Blades were mounted at
mid-chord, mid-span, and zero preset pitch, with a length or height of 1 m and
placed at 1 m diameter, giving the rotor an aspect ratio of unity. These
parameters gave the rotor a relatively high solidity $Nc/(\pi D) = 0.13$ and a
large chord-to-radius ratio $c/R = 0.28$.

The support struts were also constructed from 14 cm chord NACA 0020 profiles,
and attach the turbine to a 9.5 cm diameter shaft. A drawing of the turbine
rotor is shown in Figure~\ref{fig:rvat-drawing} and CAD models are available
from \cite{Bachant2014-RVAT-CAD}. The turbine is shown outside of the tank
installed in the test bed mounting frame in Figure~\ref{fig:rvat-photo}.

\begin{figure}
    \centering
    \begin{subfigure}{0.49\textwidth}
        \includegraphics[width=\textwidth]{unh-rvat-drawing}
        \caption{}
        \label{fig:rvat-drawing}
    \end{subfigure}
    \begin{subfigure}{0.47\textwidth}
        \includegraphics[width=\textwidth]{unh-rvat-photo}
        \caption{}
        \label{fig:rvat-photo}
    \end{subfigure}
    
    \caption{Drawing (a) and photo (b) of the UNH-RVAT vertical-axis cross-flow
        turbine. Note that the upper and lower mounting flanges (and the area of the
        shaft they cover) have been excluded in the drawing. These were included in
        the tare drag measurements for the experiments in Chapter~\ref{chap:Re-dep},
        but excluded for those in Chapter~\ref{chap:RVAT-baseline}.}
    
    \label{fig:unh-rvat}
\end{figure}


\subsection{DOE/SNL RM2}

The RM2 rotor was designed as a 1:6 scale model of that described in the RM2
``rev 0'' design report \cite{Barone2011}, with the exception of the shaft
diameter, which was scaled from the SAFL RM2 shaft \cite{Hill2014}. The hub
design is also similar to the SAFL model. A drawing of the turbine design is
shown in Figure~\ref{fig:rm2-drawing} and a photo of the RM2 installed in the
turbine test bed mounting frame is shown in Figure~\ref{fig:rm2-photo}. The
rotor has tapered blades with $c/R=0.12$ at the roots (mid-span) and $c/R=0.07$
at the tips, which uniquely gives it a medium-high solidity for a wind turbine
and low solidity for an MHK turbine. The RM2 is discussed in more detail in
Chapter~\ref{chap:RM2}.

\begin{figure}
    \centering
    
    \begin{subfigure}{0.45\textwidth}
        \centering
        \includegraphics[clip, trim=0 0.5in 0 0.6in, width=\textwidth]{rm2-drawing}
        \caption{}
        \label{fig:rm2-drawing}
    \end{subfigure}
    \begin{subfigure}{0.41\textwidth}
        \centering
        \includegraphics[width=\textwidth]{rm2-photo}
        
        \caption{}
        \label{fig:rm2-photo}
    \end{subfigure}
    
    \caption{Drawing (a) and photo (b) of the 1:6 RM2 scaled physical model cross-flow turbine.}
    \label{fig:rm2}
\end{figure}


\section{Summary and conclusions}

An upgraded test bed for measuring the performance and near-wake flows of large
laboratory scale ($O(1)$ m\textsuperscript{2} frontal area) cross-flow turbines
was developed for the UNH tow tank. To acquire adequate amounts of data and to
achieve significant lengths of steady state operation with a limited tank
length, the tow tank's linear motion, control, and data acquisition systems had
to be redesigned, rebuilt, and upgraded to allow fully automated operation,
along with higher carriage speed and acceleration. Integrating carriage,
turbine, and Vectrino probe positioning, along with synchronized data
acquisition from performance measurement, Vectrino, and motion controller
systems increased the number of tows possible per experiment by essentially an
order of magnitude, while simplifying data processing and increasing data
quality and repeatability.

Two turbine models were designed and fabricated. The UNH-RVAT turbine was
designed to be more geometrically simple, but higher solidity, while the RM2
turbine is lower solidity, with tapered blades. The differences between these
rotors provide an opportunity for direct comparison within the same experimental
setup, and will provide validation data to test numerical models' predictive
capabilities with varying levels of flow curvature (from higher $c/R$) and end
effects (blade aspect ratio and taper).
