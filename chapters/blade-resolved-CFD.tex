\chapter{On the use of blade-resolved computational fluid
    dynamics}\label{chap:CFD}

Despite the development of many simple engineering models based on blade element
momentum or vortex methods, it remains difficult to predict the performance of
cross-flow turbines (CFTs, often vertical-axis turbines or VATs) in all
cases---namely when solidity, or blade chord-to-radius ratio is high---a common
characteristic of smaller CFTs, and those designed for marine hydrokinetic (MHK)
applications. With computing power becoming evermore available and affordable,
computational fluid dynamics (CFD) has become an attractive method for
simulating cross-flow turbines since it only requires turbulence modeling, i.e.,
the physics can be described from first principles by solving the Navier--Stokes
equations using a boundary layer resolving body-fitted grid. However, CFD can be
computationally expensive when done in three dimensions, which may be necessary
in some cases.

There are many examples in the literature of 2-D cross-flow turbine simulations
with widely varying results. Balduzzi \emph{et al.}~\cite{Balduzzi2016} provides
a summary of recent computations and an attempt to standardize a methodology for
using Reynolds-averaged Navier--Stokes (RANS) to correctly predict performance
of a 2-D CFT. Howell \emph{et al.}~\cite{Howell2010}, performed both 2-D and 3-D
simulations of a high solidity cross-flow turbine using a $k$--$\epsilon$
renormalization group (RNG) turbulence model. The results from the 2-D
simulations over-predicted power coefficient, while the 3-D case matched well
with wind tunnel measurements near the tip speed ratio of maximum power. In
general, 3-D simulations are less common, but have begun to appear more
frequently recently---a testament to the progress towards higher computing
power.

An overview of 3-D blade-resolved cross-flow turbine simulations reported in the
literature is presented in Table~\ref{tab:cfd-refs}. The $k$--$\omega$ SST RANS
turbulence model is shown to be a popular choice due to its success in
predicting flows with adverse pressure gradients and
separation~\cite{Menter2003}. Higher fidelity methods that resolve the large
scales of turbulence, such as large-eddy and detached-eddy simulation have also
been used. Note that for all the studies listed, model validation was either
done for performance or wake predictions---not both---and in some case omitted
entirely.

\begin{table}
    \centering
    \begin{tabular}{c|c|c|c}
        Author & Turbulence modeling & Perf. val. & Wake val. \\ 
        \hline
        Alaimo \etal~\cite{Alaimo2015} & $k$--$\epsilon$ & N/A & N/A \\ 
        Marsh \etal~\cite{Marsh2015} & SST & \cite{Rawlings2008} & N/A \\ 
        Orlandi \etal~\cite{Orlandi2015} & SST & \cite{Akins1989,Mertens2003} & N/A \\ 
        Lam \& Peng~\cite{Lam2016} & SST RANS \& IDDES\tablefootnote{Improved delayed detached-eddy simulation.} & N/A & \cite{Tescione2014} \\ 
        Nini \etal~\cite{Nini2014} & Spalart--Allmaras & N/A & \cite{Battisti2011} \\ 
        Boudreau \& Dumas~\cite{Boudreau2015} & DDES\tablefootnote{Delayed detached-eddy simulation.} & N/A & N/A \\ 
        Li \etal~\cite{Li2013} & SST \& Smagorinsky--Lilly LES & \cite{McLaren2011} & N/A \\ 
        Howell \etal~\cite{Howell2010} & $k$--$\epsilon$ RNG\tablefootnote{Renormalization group.} & \cite{Howell2010} & N/A
    \end{tabular}
    
    \caption{Selected 3-D blade-resolved cross-flow turbine simulations reported
        in the literature, turbulence modeling employed, and performance and/or wake
        studies used for validation. Note the Li \etal study used periodic boundary
        conditions and is technically considered 2.5-D.}
    
    \label{tab:cfd-refs}
\end{table}

Modeling the boundary layer flows on cross-flow turbine blades is very
challenging due to the dynamically changing inflow velocity and angle of
attack---which often exceeds static stall values and causes dynamic stall.
Furthermore, the ability to predict the occurrence and interdependence of
boundary layer transition to turbulence and separation can have dramatic
influence on the blade loading and therefore the predicted turbine power output.
These challenges present significant obstacles to the prospect of using CFD to
replace wind tunnel or tank testing of physical models.

To date, little computational work has been done to attempt to design arrays of
CFTs, despite their prospects for closer spacing compared with axial-flow
turbines (AFTs). For example, Araya \etal~\cite{Araya2014} modeled the flow
through a VAT array using ``leaky Rankine body'' potential flow singularities,
which was able to rank relative---though not absolut---performance of array
configurations. Goude and Agren~\cite{Goude2010} used a 2-D vortex method to
simulate a farm of cross-flow turbines, though this was not validated with
experiments. Durrani \emph{et al.}~\cite{Durrani2011} used 2-D CFD to model a
group of cross-flow turbines, observing higher power output for a staggered
configuration, but also did not compare with experimental results. Giorgetti
\etal~\cite{Giorgetti2015} took a similar approach for 2-D array analysis using
turbine pairs inspired by Dabiri~\cite{Dabiri2011}, but again experimental
validation was not performed. Li and Calisal \cite{Li2010} used a 3-D vortex
line method to show mutually improved power output from two adjacent turbines,
though the simulations over-predicted the effects compared with experiments.
Antheaume \emph{et al.}~\cite{Antheaume2008} used a blade element approach
coupled with a 3-D RANS solver to also show how close spacing can improve power
output of CFTs.

In this study we set out to model the performance and near-wake of the high
solidity University of New Hampshire Reference Vertical Axis Turbine (UNH-RVAT)
using a Reynolds-averaged Navier--Stokes numerical model. Though studies in the
literature generally focus on predicting the turbine loading and the local blade
boundary layer, we seek to model both this and the larger scale flow produced by
the rotor. The near-wake of this particular turbine is of interest since it has
been shown experimentally that the near-wake's momentum and energy transport
processes are dominated by vertical advection \cite{Bachant2015-JoT}. It
logically follows that a 2-D simulation, which omits the vertical dimension,
would not correctly predict wake recovery. However, it is of interest to
determine how wrong a 2-D model may be, since the computational feasibility is
attractive.

It is assumed that if the turbine power output and near-wake predictions from
the numerical model match those of the experiments, the flow field can be
inspected in greater detail, i.e., that the experiments will have been
``interpolated.'' This will give us access to unmeasured quantities, e.g.,
pressure, and allow us to see where the dominant flow structures originate. This
will ultimately help develop and evaluate low-order wake generator models for
use in turbine array modeling. In summary, the questions we hope to answer here
are:

\begin{enumerate}

    \item Can or should 2-D RANS be used for individual turbine and/or array
    design?

    \item How accurately can 3-D RANS predict performance?

    \item Can 3-D RANS ``interpolate'' the experimental results and provide
    insight to develop new low-order wake generators to represent CFTs?

    \item Does 3-D RANS realize the correct proportions of wake recovery
    mechanisms, i.e., are the 3-D blade-resolved results a good ``target'' for
    those a low-order model should produce?

\end{enumerate}


\section{Turbine description}

In this study we sought out to predict the near-wake of the UNH-RVAT turbine,
for which experiments were performed in Chapter~\ref{chap:RVAT-baseline} and
Chapter~\ref{chap:Re-dep}. The turbine was simulated at a tip speed ratio
$\lambda=1.9$, which corresponds to the maximum power coefficient, and a turbine
diameter Reynolds number $Re_D \approx 10^6$, which corresponds to the
$Re$-independent state for both performance and near-wake characteristics, as
determined from experimental measurements \cite{Bachant2014,
    Bachant2016-Energies}.

Turbine geometry was prepared or ``cleaned'' for CFD by removing details
determined to be unnecessary, e.g., screw heads and axial shaft grooves, which
would complicate the meshing. A drawing of the physically and numerically
modeled geometries is presented in Figure~\ref{fig:cfd-cad}.

\begin{figure}
    \centering

    \includegraphics[clip, trim=0 1in 0 1in, width=0.8\textwidth]{cfd-cad}

    \caption{CAD drawings of the UNH-RVAT cross-flow turbine as designed (left)
        and cleaned for simulation (right).}

    \label{fig:cfd-cad}
\end{figure}


\section{Numerical models}

The flow field was modeled using the Reynolds-averaged Navier--Stokes equations,
employing two different turbulence models---Menter's $k$--$\omega$ SST
\cite{Menter1994} and the Spalart--Allmaras (SA) one equation model
\cite{Spalart1992}. Both closures use the eddy-viscosity approach---the SA
employing a single additional scalar transport equation and the SST two. The SST
model was chosen due to its prominence in the literature for simulating
separating flows, which we assumed to be present in the current problem in the
form of dynamic stall. The SA model was shown by Ferreira et al.
\cite{Ferreira2007} to match experimental particle image velocimetry (PIV)
results for a CFT in dynamic stall, though this was a somewhat low Reynolds
number case ($5 \times 10^4$). Further justification for using the SA model for
this case comes from Crivellini and D'Alessandro \cite{Crivellini2014}, where
they successfully modeled the laminar separation bubble and subsequent boundary
layer transition to turbulence at Reynolds numbers similar to ours.


\subsection{Computational mesh}

The computational domain was a rectangular volume 3.66 m long, 3.66 m wide, and
2.44 m tall (for 3-D), with the turbine located 1.52 m from the inlet, and
centered vertically with a vertical axis, designed to match the tow tank
dimensions for comparison with previous experiments. The rotor geometry was
located at the center of a cylindrical sliding mesh interface, which rotated at
a mean tip speed ratio $\lambda=1.9$ with a sinusoidal oscillation at the blade
passage frequency---with and amplitude of 0.19 and the first peak at 0.7
radians---to mimic the experimental data. The 2-D mesh overview is shown in
Figure~\ref{fig:2d-br-mesh} and the blade mesh is shown in
Figure~\ref{fig:blade-mesh}.

Meshes were generated using \textit{OpenFOAM}'s \textit{blockMesh} and
\textit{snappyHexMesh} utilities. Mesh topology consists of a background
hexahedral mesh, which is refined in all three directions by a factor of 2 in a
rectangular region containing the turbine and near-wake (0.9 m upstream, 1.3 m
downstream, $\pm 0.9$ m cross-stream, and $\pm 0.8$ m vertically). Cells
adjacent to the turbine shaft and struts are refined by a factor of 4, while
cells adjacent to the blades are refined by a factor of 6. To capture the
boundary layer, 20 layers are added next to the blades with an expansion ratio
of 1.2. Overall mesh refinement is controlled by a single parameter---the number
of cells in the streamwise direction, $N_x$.

\begin{figure}
    \centering
    
    \includegraphics[width=0.8\textwidth]{2d-br-mesh}

    \caption{Overview of the 2-D computational mesh.}

    \label{fig:2d-br-mesh}
\end{figure}


\begin{figure}
    \centering
    
    \includegraphics[width=0.7\textwidth]{2d-blade-mesh-closeup}

    \caption{Detailed view of the 2-D computational mesh near the blades.}

    \label{fig:blade-mesh}
\end{figure}


\subsection{Solver}

Simulations were run using the \textit{pimpleDyMFoam} solver from the
open-source finite volume CFD package \textit{OpenFOAM}, version 2.3.x.
\textit{pimpleDyMFoam} uses a hybrid PISO-SIMPLE algorithm for pressure-velocity
coupling and is compatible with dynamic meshes.

\subsection{Initial and boundary conditions}

Initial and boundary conditions were set to match those of the tow tank as well
as possible. The velocity at the bottom and side walls was fixed to 1 m/s to
match the tow tank case, while the top boundary condition was a slip velocity
condition. Note that in 2-D the top and bottom boundary conditions are
``empty,'' which is an OpenFOAM convention to indicate two-dimensionality.


\section{Model verification}

Both the $k$--$\omega$ SST and Spalart--Allmaras RANS model cases were verified
for convergence of the turbine mean power coefficient with respect to grid
spacing and time step using 2-D simulations. The grid topology was fixed, but
the number of cells per domain length was scaled proportionally, maintaining the
same background mesh cell aspect ratio. Results for this parameter sweep are
shown in Figure~\ref{fig:2d-br-verification}, from which the final number of
streamwise grid points $N_x = 70$ was chosen. This corresponds to a total cell
count of approximately $5 \times 10^4$ for the 2-D cases and 16 million for the
3-D case.

\begin{figure}
    \centering

    \includegraphics[width=0.9\textwidth]{2d-br-verification}

    \caption{Time step (left) and grid size (right) dependence for the 2-D case
        with both the SST and SA turbulence models. Time step dependence was carried
        out with $N_x=70$ and grid size dependence with the time steps annotated for
        each turbulence model.}

    \label{fig:2d-br-verification}
\end{figure}

Time step dependence was studied using the $N_x=70$ grid, the results from which
are shown in Figure~\ref{fig:2d-br-verification}. It was seen that the
Spalart--Allmaras model converged well with decreasing time step, leading to a
final time step of 0.001 s. The results from the SST model show a local minimum
at $\Delta t = 0.002$ s, with some divergence for smaller time steps. The local
minimum was chosen as the final time step to run the simulations. Note that the
SST model's convergence behavior may be due to its specific implementation in
\textit{OpenFOAM}, and not indicative of the nature of the model equations.
Verification studies for CFTs with this level of detail in the literature are
not common, though the final time step is comparable to others
\cite{Balduzzi2016}.


\section{Results}

Computations for the 3-D cases were run on 192 processes and took on the order
of 1,000 CPU hours per second of simulated time. The 2-D simulations were run on
a single processor and took on the order of one CPU hour per second.


\subsection{Performance prediction}

Predictions for both the mean rotor power and drag coefficients are shown in
Figure~\ref{fig:br-cfd-perf-bar-chart}. In general, the 2-D CFD cases both
overpredict turbine loading, which is likely due to their increased blockage
ratio, unresolved blade end effects, and lack of blade support struts.

The 3-D simulations fair better at predicting the experimental measurements,
with the Spalart--Allmaras model performing relatively better. The apparent
overprediction of rotor drag coefficient could be an effect of the experimental
procedure, where the ``tare drag'' from the turbine mounting structure was
measured without a turbine installed, then subtracted in post-processing.
Technically the local flow field will have changed, and hence the tare drag will
change as well.

\begin{figure}
    \centering

    \includegraphics[width=0.95\textwidth]{br-cfd-perf-bar-chart}

    \caption{Power (left) and drag (right) coefficient predictions from
        experiments and each numerical model.}

    \label{fig:br-cfd-perf-bar-chart}
\end{figure}


\subsection{Wake characteristics}

Visualizing the complicated wake generated by the turbine is presented for the
2-D and 3-D Spalart--Allmaras cases in Figure~\ref{fig:br-vorticity-3d} and
Figure~\ref{fig:br-vorticity-3d}, respectively. It can be seen how the upstream
blade---as it turns back into the streamwise direction---is shedding a large
amount of spanwise vorticity due to the separated flow. In the 3-D case, strong
tip vortices are also present, which trace the ``contracting'' wake flow on the
$-y$ side of the turbine associated with the induced vertical velocity field.
The 3-D dynamic stall vortex also shows asymmetry about the $x$--$y$ mid-rotor
plane; once again highlighting the importance of three-dimensional effects on
wake dynamics.

\begin{figure}
    \centering
    
    \includegraphics[width=0.75\textwidth]{2D_vorticity_SA_964}

    \caption{Instantaneous vorticity contours (at $t=9.64$ s) computed for the
        2-D Spalart--Allmaras case.}

    \label{fig:br-vorticity-2d}
\end{figure}

\begin{figure}
    \centering

    \includegraphics[width=0.75\textwidth]{3D_vorticity_SA_964_10-threshold}

    \caption{Iso-vorticity contours (at $t=9.64$ s) colored by the streamwise
        component of vorticity for the 3-D Spalart--Allmaras case.}

    \label{fig:br-vorticity-3d}
\end{figure}

Mean velocity profiles at one turbine diameter downstream are shown in
Figure~\ref{fig:br-cfd-profiles}. The 2-D results suffer from a blockage
mismatch, i.e., keeping the proximity of the walls constant increases the
blockage ratio. The 3-D results, however, show good agreement with the
experiments.

\begin{figure}
    \centering

    \includegraphics[width=0.95\textwidth]{br-cfd-profiles}

    \caption{Mean velocity (left) and turbulence kinetic energy (right) profiles
        at $x/D=1$ from 2-D simulations, 3-D simulations ($z/H=0$), and experiments
        \cite{Bachant2015-JoT}.}

    \label{fig:br-cfd-profiles}
\end{figure}

Turbulence kinetic energy profiles are also shown in
Figure~\ref{fig:br-cfd-profiles}. The turbulence kinetic energy is calculated as
\begin{equation}
    k = k_{\mathrm{RA}} + \frac{1}{2} \left(
    \overline{U^\prime}^2 +
    \overline{V^\prime}^2 +
    \overline{W^\prime}^2 \right),
    \label{eq:k}
\end{equation}
where $U^\prime = U - \overline{U}$ and $k_{\mathrm{RA}}$ is the kinetic energy
calculated by the turbulence model, which is zero for the SA model. Note that
the statistics of the Reynolds-averaged velocity are calculated from a time
series that has been downsampled to 50 Hz. It is assumed that this is far enough
from the blade passage frequency that differences from the variance in the
original velocity will be negligible.

Both Spalart--Allmaras cases do a poor job predicting the turbulence kinetic
energy in the flow, since it must be resolved as variance in the velocity field.
The 2-D SST model does a good job predicting the peak in $k$ at $y/R=-1$, though
is missing the smaller peak at $y/R=-1$. This is once again likely due to
blockage issues, where local tip speed ratio is decreased, increasing the
blades' instantaneous angle of attack at this location on the downstream
passage. In contrast, the 3-D SST model predicts the $+y$ peak in turbulence
kinetic energy very well, though the $-y$ peak magnitude is overpredicted by
about 30\%. We also see some smearing of $k$ across the center of the rotor,
which is likely due to exaggerated levels of the turbulent eddy viscosity.


\subsubsection{Mean velocity in three dimensions}

In order to visualize the mean velocity field, vector arrows for the mean
cross-stream and vertical components are superimosed on top of contours of the
streamwise component at $X/D=1$ in Figure~\ref{fig:br-cfd-mean-velocity}. Both
CFD models predict the general structure of the mean velocity well, though the
SA case has slightly larger vertical mean flows, which could be due to stronger
tip vortex generation, or lower diffusivity compared with the SST model.
Additional discrepancies between CFD and experiments may be due to the top slip
boundary condition versus the experiment's free surface.

\begin{figure}
    \centering
    \begin{subfigure}[b]{\textwidth}
       \centering

        \includegraphics[clip, trim=0 0.1in 0.3in 0.2in,
        width=0.72\textwidth]{RVAT-Re-dep_meancontquiv_10}

       \caption{Mean velocity field at $x/D=1$ from experiments
           \cite{Bachant2016-RVAT-Re-dep}.}

       \label{fig:br-cfd-meancontquiv-exp}
   \end{subfigure}

   \begin{subfigure}[b]{\textwidth}
       \centering

        \includegraphics[clip, trim=0 0.2in 0 0.15in,
        width=0.8\textwidth]{meancontquiv_kOmegaSST}

       \caption{Mean velocity at $x/D=1$ computed by the 3-D SST model.}

       \label{fig:meancontquiv-SST}
   \end{subfigure}

   \begin{subfigure}[b]{\textwidth}
       \centering

        \includegraphics[clip, trim=0 0.2in 0 0.15in,
        width=0.8\textwidth]{meancontquiv_SpalartAllmaras}

       \caption{Mean velocity at $x/D=1$ computed by the 3-D SA model.}

       \label{fig:meancontquiv-SA}
   \end{subfigure}

    \caption{Mean velocity from experiments and 3-D CFD cases. Solid gray lines
        indicate turbine frontal area and dashed lines indicate experimental
        measurement plane.}

    \label{fig:br-cfd-mean-velocity}
\end{figure}


\subsubsection{Turbulence kinetic energy contours}

Turbulence kinetic energy contours for the experimental measurements and each
CFD case at $x/D=1$ are presented in Figure~\ref{fig:br-cfd-kcont}. As seen in
the profiles in Figure~\ref{fig:br-cfd-profiles}, the SA model is resolving very
little of the flow unsteadiness. In contrast, the SST model does a good job
predicting the locations and magnitudes of various peaks in $k$. These are
generated along the top of the turbine via tip vortex shedding, and the $-y$
side of the turbine via dynamic stall. We do however see the smearing effect
from the dynamic stall vortex centered around $z/H=0$, which is likely more of
an issue with wake evolution rather than wake generation.

\begin{figure}
    \centering
    \begin{subfigure}[b]{\textwidth}
        \centering

        \includegraphics[width=0.95\textwidth]{RVAT-Re-dep_k_contours_10}

        \caption{Turbulence kinetic energy at $x/D=1$ from experiments
            \cite{Bachant2016-RVAT-Re-dep}.}

        \label{fig:kcont-exp}
    \end{subfigure}

    \begin{subfigure}[b]{\textwidth}
        \centering

        \includegraphics[width=0.95\textwidth]{figures/kcont_kOmegaSST}

        \caption{Turbulence kinetic energy at $x/D=1$ computed by the 3-D SST
            model.}

        \label{fig:kcont-SST}
    \end{subfigure}

    \begin{subfigure}[b]{\textwidth}
        \centering

        \includegraphics[width=0.95\textwidth]{figures/kcont_SpalartAllmaras}

        \caption{Turbulence kinetic energy at $x/D=1$ computed by the 3-D SA
            model.}

        \label{fig:kcont-SA}
    \end{subfigure}

    \caption{Turbulence kinetic energy from experiments and 3-D CFD cases. Solid
        black lines indicate turbine frontal area.}

    \label{fig:br-cfd-kcont}
\end{figure}


\subsubsection{Momentum recovery}

To get an overall idea of the wake recovery predicted by each model, we
rearrange the streamwise component of the Navier--Stokes equation to isolate
$\partial U / \partial x$---following Bachant and
Wosnik~\cite{Bachant2015-JoT}---and compute each term at $X/D = 1$ to compare
with the experimental results.

We use the RANS models' eddy viscosity to calculate the turbulent transport via
\begin{equation}
    \text{Turb. trans.} = \nu_t \nabla^2 \vec{U},
    \label{eq:turb-trans}
\end{equation}
which is a different approach from those taken on the experiments, where
Reynolds stresses were measured, but $x$-derivatives were not:
\begin{equation}
    \text{Turb. trans. (exp.)} =
    -\left(
    \frac{\partial}{\partial y} \overline{u^\prime v^\prime}
    +
    \frac{\partial}{\partial z} \overline{u^\prime w^\prime}
    \right).
\end{equation}
As such, we should not be surprised if the CFD models predict higher levels of
turbulent transport than the experiments.

Normalized weighted averages for each recovery term at $x/D=1$ are computed and
multiplied by the cross-sectional area of the measurement plane, or the channel
width in the 2-D cases. Results are shown in a bar chart in
Figure~\ref{fig:br-cfd-recovery}. Consistent with the relatively large Reynolds
number regime, viscous transport is essentially negligible compared with other
mechanisms. Cross-stream advection---or the tendency of streamlines to diverge
and reduce the streamwise momentum---produces a negative effect for all cases,
though the 3-D SST model predicts significantly lower values. Vertical advection
is by definition zero for the 2-D cases. The 3-D cases show varying
results---with the SST model overpredicting and SA underpredicting the vertical
velocity's effect on replenishing streamwise momentum.

Turbulent transport and streamwise pressure gradient terms show the largest
discrepancy between results. The 3-D SST case, despite doing a good job
predicting turbulence kinetic energy, significantly overpredicts the turbulent
transport term, while other CFD cases are comparable with the experiment. This
seems to be balanced by a large adverse pressure gradient, which is also present
to a smaller degree in the 3-D SA case. Interestingly, in contrast, both 2-D CFD
cases create a wake where the pressure gradient is acting to accelerate the flow
at $x/D=1$. Unfortunately, pressure data were not acquired from the experiment,
though the 3-D delayed detached eddy simulation (DDES) of Boudreau and Dumas
\cite{Boudreau2015} concur with the adverse pressure gradient condition.

\begin{figure}
    \centering

    \includegraphics[width=0.9\textwidth]{br-cfd-mom-bar-chart}

    \caption{Weighted sum normalized momentum recovery terms for each CFD case
        and experiments\cite{Bachant2016-RVAT-Re-dep} at $x/D=1$.}

    \label{fig:br-cfd-recovery}
\end{figure}


\section{Conclusions}

A cross-flow turbine was modeled using the $k$--$\omega$ SST and
Spalart--Allmaras (SA) Reynolds-averaged Navier--Stokes turbulence models to test
their abilities to predict turbine performance and near-wake dynamics. It was
observed that when modeled in 2-D, the performance is over-predicted compared to
the data from the tow tank experiments, which was expected due to omission of
blade end effects, support strut drag, and increased blockage. Vertical (or
axial) wake dynamics were unresolved in the 2-D model, despite being identified
as a significant contributor to streamwise wake recovery, which casts doubt on
the 2-D model's applicability as a tool to study array spacing effects.

The 3-D blade-resolved RANS simulations predicted turbine performance and
near-wake quite well, where the SA model results were closest to the
experimentally measured performance. Both the SST and SA models did a good job
predicting the mean velocity field at one turbine diameter downstream, but the
SST model was more effective at predicting turbulence kinetic energy, since it
is solved for in the turbulence model equations. Streamwise momentum recovery
terms were computed from the CFD results over an entire cross-section of the
domain at $x/D=1$. Values for transport due to the mean pressure gradient and
turbulent fluctuations varied a lot between the two turbulence models. Both
models, however, we able to at least qualitatively resolve the vertical velocity
field, which will be crucial to predicting the performance of closely spaced
CFTs.

The effectiveness at predicting the mean velocity field gives credibility to the
prospect of using the computed flow field to interpolate the experimental
results, such that the CFD results can be used as a target for a low-order wake
generator or force parameterization. These results may also help develop new tip
loss corrections for blade element type models, which currently only exist for
axial-flow rotors, since we have access to the pressure and shear forces over
the entire blade surface, which were not measured experimentally. Ultimately,
however the computational cost of 3-D simulations---about 10,000 CPU hours per
operating case---may be too expensive to be used for CFT engineering work,
especially considering the uncertainty involved compared with physical model
studies.


\section{Acknowledgments}

Thanks to Vincent S. Neary and Sandia National Labs for the use of their Red
Mesa high performance computing cluster to run the simulations.

