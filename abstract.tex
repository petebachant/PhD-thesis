\begin{Abstractpage}

\setlength{\baselineskip}{1.5\baselineskip}
{

Cross-flow (often vertical-axis) turbines (CFTs), despite being thoroughly
investigated and subsequently abandoned for large scale wind energy, are seeing
renewed interest for smaller scale wind turbine arrays, offshore wind, and
marine hydrokinetic (MHK) energy applications. Though they are similar to the
large scale Darrieus wind turbines, today's CFT rotors are often designed with
higher solidity, or blade chord-to-radius ratios, which makes their behavior
more difficult to predict with numerical models. Furthermore, most experimental
datasets used for numerical model validation were acquired with low solidity
rotors.

An experimental campaign was undertaken to produce high quality open datasets
for the performance and near-wake flow dynamics of CFTs. An automated
experimental setup was developed using the UNH's tow tank to measure the
performance and near-wake of CFTs at ~1 m scale.

Two turbines were investigated---one high solidity (dubbed the UNH Reference
Vertical-Axis Turbine or UNH-RVAT) and one medium-to-low solidity, which was a
scaled model of the US Department of Energy and Sandia National Labs' Reference
Model 2 (RM2) cross-flow MHK turbine. A baseline performance and near-wake
dataset was acquired for the UNH-RVAT, which revealed that the relatively fast
wake recovery observed in vertical-axis wind turbine arrays could be attributed
to the mean vertical advection of momentum and energy, caused by the unique
interaction of vorticity shed from the blade tips.

The Reynolds number dependence of the UNH-RVAT was investigated, the results
from which indicated that independence was achieved at a rotor diameter Reynolds
number $Re_D \sim 10^6$. A similar study was undertaken for the RM2, with
similar results. The wake of the RM2 also showed the significance of mean
vertical advection on wake recovery, though the lower solidity made these
effects weaker than for the UNH-RVAT.

Blade-resolved Reynolds-averaged Navier--Stokes (RANS) computational fluid
dynamics (CFD) simulations were performed to assess their ability to model
performance and near-wake of the UNH-RVAT. The 2-D simulations were a poor
predictor of both the performance and near-wake. 3-D simulations faired much
better, but their high computational expense precludes their use for array
analysis.

Finally, an actuator line model (ALM) was developed to reduce the cost of 3-D
CFD simulations of CFTs, since previously, the ALM had only been investigated
for a low Reynolds number 2-D CFT. Despite retaining some disadvantages of the
lower fidelity blade element momentum and vortex methods, the ALM was able to
predict the performance reasonably well. Near-wake results matched some of the
important qualitative flow features, which warrants further development.


}


\end{Abstractpage}
